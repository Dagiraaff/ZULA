\chapter{Einleitung}

\section{Maris Putnins}
Maris Putnins wurde 1950 in Valmiera, Lettland geboren ist als `Schauspieler, Puppenspieler und Bühnenbildner' tätig (vgl.1) und hat sowohl als Drehbuchautor als auch als Produzent bei mehreren Filmen, Kurzfilmen und einer Fernsehserie mitgewirkt, die hauptsächlich in Lettland bekannt  sind, (vgl. 2) darunter „Die kleinen Bankräuber“ (Originaltitel: „Mazie laupitaji“ – 2009), der von einem kleinen Jungen handelt, der plant, zusammen  mit seiner Schwester die Bank zu überfallen, die soeben seine Eltern aus der neuen Wohnung verwiesen hat, weil sein Vater seinen Arbeitsplatz verloren hat. (siehe 3) 
Sein Buch `Die wilden Piroggenpiraten' wurde 2013 für den Deutschen Literaturpreis nominiert und ist für Kinder ab 8 Jahren geeignet.(4)

\section{Formulierung der Fragestellung}
Im Folgenden wird das Buch „Die Piroggenpiraten“, literarisch analysiert. Nach dem ersten Eindruck folgt ein Einblick in den Handlungsablauf, die Hauptprotagonisten werden charakterisiert und wichtige Schauplätze aufgezeigt. Bei der Strukturanalyse wird der Handlungsaufbau dargelegt, Spannungskurven und Erzähltechnik erläutert. Bei der Sprachanalyse wird speziell auf die Semantik im Zusammenhang mit Lebensmitteln und Nautik, sowie den Erzählstil und (geschichtliche) Anspielungen eingegangen. Nach einer Erklärung wie man das Buch im Unterricht verwenden kann und welche Stellen im Lehrplan für eine Verwendung relevant sein können, schließt die Arbeit mit Unterrichtsbeispielen. Der Einfachheit halber werden die Figuren auch als „Personen“ bezeichnet oder in einer anderen Form menschlich benannt, z.B. Mädchen, Mann, Kind usw. da es sich um personifizierte Lebensmittel handelt, die sich ebenso wie Menschen bewegen, denken, fühlen und handeln. Falls nicht anders angegeben wird aus dem Buch `Die Piroggenpiraten'(5) von Maris Putnins zitiert.

\section{Erster Eindruck}
Das Titelbild zeigt drei Personen, die an der Spitze eines Schiffes stehen, das von Möwen umkreist wird. Die weibliche Person ist rothaarig und sommersprossig, hält sich mit der Linken an einem Seil fest und hat in der rechten Hand ein Schwert. Ein schlanker junger Mann trägt eine rote Mütze und wird von einem korpulenten Herrn verdeckt, der sowohl eine Kanone als auch ein Messer bei sich hat. Die zwei männlichen Personen stehen hinter dem Mädchen, alle Drei blicken nach rechts, in die Richtung in die sie sich bewegen. Aufgrund der Art ihrer Kleidung, der mitgeführten Accessoires und dem sie umgebenden Meer scheint es sich um Piraten zu handeln. Der Bildaufbau mit der Dynamik nach Rechts erzeugt den Eindruck des Aufbruchs, des Tatendrangs und in Verbindung mit der Mimik der Dargestellten, der Vorfreude. Womöglich sind sie gerade dabei auf einer Insel zu landen um einen Schatz zu suchen oder am Festland eine Stadt zu überfallen, zumindest lassen die Möwen darauf schließen, dass sie sich in Küstennähe befinden.
Der Titel `Die wilden Piroggenpiraten' bestätigt die Annahme, dass es sich um Seeräuber handelt und erzeugt zusammen mit dem Untertitel Neugier, da von einer `entführten Mohnschnecke' die Rede ist und man dieses Gebäck zunächst nicht mit dem Titelbild in Verbindung bringen kann. Erst in Verbindung mit dem Klappentext auf der Rückseite erschließt sich, dass es sich bei der Mohnschnecke, ebenso wie bei `eine(r) wilde(n) Pirogge', `ein(em) Hörnchen' und `ein(em) Eclair' um personifizierte Gebäckstücke handelt. Aufgrund der äußere Aufmachung, den erwähnten, scheinbar lebendigen Teigwaren, den Schauplätzen - Piratenschiff, Kloster und Kerker- und dem ungewöhnliche Schlachtruf `Macht sie zu Semmelbröseln!' scheint es sich um ein fantastisches, spannendes und wildes Piratenabenteuer zu handeln.

\chapter{Die wilden Piroggenpiraten}

\section{Inhaltsanalyse}

\subsection{Handlung}

\subsubsection{Überblick}
Mohnschnecke fährt mit Eclair, einem Angestellten ihres Vaters, nach Sankt Krokant um zu einem Ball des Grafen Napoleon zu gehen. Allerdings wird dort sie bei einem Yachtausflug mit Hörnchen von wilden Piroggen entführt. (Kapitel 1-3) Während sie sich mit der Zeit auf dem Schiff der Seeräuber einlebt, versuchen Eclair und Hörnchen unabhängig voneinander das Mädchen zu retten.  Eclair kommt zunächst in ein Pelmenidorf, lernt dort Otto kennen und macht sich mit seiner Hilfe auf einer  Brigantine auf die Suche nach Mohnschnecke. Als die beiden schiffbrüchig werden und einem Reispiroggenschiff aufgenommen werden, segeln sie zusammen mit ihren Rettern nach Trüffelgar. Mohnschnecke, die Hauptfigur lebt sich auf dem Piroggenschiff ein, wird nach Mordans Abschied ins Kloster Kapitän und führt die Piroggen in der Schlacht von Trüffelgar an. (Kapitel 43) Aber sie sehnt sich trotz der schönen Zeit bei den Piroggen nach dem Ball, von dem sie schon seit ihrer Kindheit träumt. Verraten von Halbpirogg und Hörnchen, die zusammen Rache geplant haben, wird sie wegen ihrer Vergehen als Piroggenanführerin auf Sankt Krokant verhaftet. Eclair, Otto und die Mannschaft versuchen sie mit allen Mitteln zu befreien, letztendlich bringt der aus dem Kloster zurückgekehrte Mordan jedoch die glückliche Wende.

\subsubsection{Ausgangs- und Endsituation}
Die Ausgangssituation wird im Prolog und in den Kapiteln eins bis drei beschrieben. Zunächst wird von den Piroggen, ihrer Herkunft und ihren Eigenschaften erzählt und von der Handelsstadt Murseille berichtet. Murseille ist der Ausgangsort von Mohnschnecke und Eclair. Sie wohnt dort bei ihren wohlhabenden Eltern, die ein Mohngeschäft betreiben und er arbeitet dort für ihren Vater. Das Mädchen möchte einen Ball besuchen von dem sie schon lange träumt und fragt ob er sie nach Sankt Krokant rudern könne. Dort treffen sie auf Hörnchen, der das Mädchen einlädt einen Ausflug auf seiner Yacht zu machen. Alles ändert sich, als sie von wilden Piroggenpiraten entführt wird und Eclair und Hörnchen sich auf die Suche nach ihr machen.
\\
Am Ende muss Mohnschnecke, (Kapitel 59 bis Epilog) die nun Käptn Mohnschnecke ist, nicht mehr von den Piroggen, sondern aus dem Gefängnis der Stadt Murseille befreit werden. Dorthin war sie gebracht worden nachdem sie von Halbpirogg und Hörnchen verraten worden war und wartet nun auf die Todesstrafe durch Verbrennen. Eclair versucht sie zusammen mit der Mannschaft der Speckkugel zu befreien, doch erst als Mordan plötzlich auftaucht, gelingt es allen beteiligten Piraten zu fliehen. Zurück auf dem Schiff, wird Halbpirogg verbannt, Mordan wieder Käptn und Mohnschnecke und Eclair verlassen gemeinsam die Piroggen. Auf Graf Napoleons Frühlingsball macht Eclair dem Mädchen einen Heiratsantrag und die beiden sind nun mit „einer kleinen aber treuen Schwarzpiroggenmannschaft“ unterwegs. (S.643) Mohnschneckes Vater wird noch reicher, als er unerwartet, möglicherweise von seiner Tochter, mit einer großen Kamelherde beschenkt wird, die „Mohnsaat aus Damaskus“ (S.645) geladen hat.
Hörnchen heiratet die Zimtschnecke, die er auf dem Markt von Djadida für Mohnschnecke gehalten hatte (S.644) und Käptn Mordan entdeckt mit seiner Mannschaft einen neuen Kontinent. (S.646)


\subsubsection{Handlungsbestimmende Ereignisse}

Jeder der drei Hauptprotagonisten, sowie Halbpirogg und Käptn Mordan erleben Ereignisse die die Geschichte grundlegend beeinflussen. Das Schicksal von Mohnschnecke hat Auswirkungen auf den gesamten Handlungsverlauf: Ohne ihre Entführung könnte sich die Geschichte nicht so Anders von der Ausgangssituation entwickeln: keiner der Beteiligten hätte einen Grund nach ihr zu suchen. Die Neuerungen an Bord und ihre persönliche Entwicklung beeinflussen Halbpirogg negativ und Mordan positiv. Mordans Heiratsantrag in Kapitel 31 führt dazu, dass Käptn Mordan ins Kloster geht, sie Kapitän auf der Speckkugel wird, damit aber auch ihre gute Stimmung auf dem Schiff kippt und sie beginnt sich und ihr Handeln zu hinterfragen, was auf längere Sicht zu ihrem emotionalen Tiefpunkt im Gefängis führt. (Kapitel 57) Halbpiroggs andauernde Unzufriedenheit gegenüber Mohnschnecke führt schließlich zu seinem Racheplan, sodass er zusammen mit Hörnchen, das Mädchen verrät und festnehmen lässt. (Kapitel 55) Die zunächst kleine Auseinandersetzung mit den Spuniern in Kapitel 26, schaukelt sich über Kapitel 28, der „Flucht à la Venezia“, bis zur „Schlacht von Trüffelgar“(Kapitel 43) hoch. Die Mannschaft der Speckkugel und auch Kapitän Li, der  das Reispiroggeschiff auf dem sich Eclair befindet befehligt, erfahren von den Pilzpiroggen, dass sich alle Piroggen zu einer Schlacht zusammenfinden (Kapitel 32 und 39)
Eclair trifft nach der Verbannung von seiner Wohn- und Arbeitsstelle in Murseille auf ein Pelmenidorf (Kapitel 5), dort wiederum lernt er Otto kennen, mit dem zusammen er eine Brigantine repariert, die ihm die Möglichkeit eröffnet auf See nach Mohnschnecke zu suchen. Durch den Sturm in Kapitel 22 wird zwar ihr Boot zersört, dadurch werden sie jedoch von den Reispiroggen aufgenommen und segeln mit zur Schlacht bei Trüffelgar. Obwohl Eclair jetzt direkt vor der Gesuchten steht, schickt sie ihn und seinen Freund nach Hause. Hätte sein Begleiter und Freund Otto, der Pelmen, nicht schon vor der Reise den dringenden Wunsch ein echter Pirogg zu werden, würde ihn Eclair nicht zum Piroggenbackofen begleiten, (Kapitel 49) weswegen Otto auf der Speckkugel arbeiten kann (Kapitel 52) und somit würde Eclair wohl nicht erfahren, dass Mohnschnecke in Schwierigkeiten ist (Kapitel 56). Ohne diesen Umstand könnte er sich ebensowenig mit der Mannschaft der Speckkugel verbünden, um Mohnschnecke zu retten. (Kapitel 56) Hörnchen wird erst von seinem Vater dazu gedrängt nach Mohnschnecke zu suchen (Kapitel 10). Die Verkleidung, die er und Zwieback bei ihrer Flucht von der Burg des Vicomte de Brinson, (Kapitel 25) von Alalie erhalten, hilft ihnen zwar über die Grenze nach Wursterreich, lässt sie aber als „Spione“ auffliegen, sodass sie ins Gefängnis kommen (Kapitel 30) Die Bekanntschaft mit der Knofikarawane (Kapitel 14) führt zwar dazu, dass der Knofiälteste ihm zu einem Freispruch verhilft, der Vicomte de Brinson legt jedoch Einspruch ein: es kommt zu einem Turnier. (Kapitel 33) Nachdem Hörnchen das Turnier gewonnen hat (Kapitel 35), fährt er mit Zwieback nach Djadida, wo sie eine vermummte, vermeintliche Mohnschnecke kaufen (Kapitel 38), die sich zu Hause als Zimtschnecke herausstellt. (Kapitel 50) Die dadurch erfahrene Ablehnung in seiner Heimatstadt (Kapitel 50) macht ihn empfänglich für Halbpiroggs Racheplan. (Kapitel 54) Mordan erfährt im Kloster von Mohnschneckes Verhaftung und dem nahenden Ende der Piroggenherrschaft (Kapitel 61), sodass er schließlich als Mönch die ausschlaggebende Rolle bei ihrer Befreiung spielt. (Kapitel 64)


\section{Charaktere}
\subsection{Hauptfiguren}

\subsubsection{Mohnschnecke}

Die Hauptfigur ist „eine an Jahren gänzlich junge und schöne Mohnschnecke“ (S.13). Ein „sehr sentimental(es)“ Mädchen (S.15), das  zwar aus gutem Hause kommt (S.14) „viele Romane (liest)“ und dementsprechend gebildet ist  (S.15), sich ihrer gesellschaftlichen Stellung aber durchaus bewusst ist: Als Eclair sie mit „Guten Morgen Mademoiselle!“ begrüßt und „sich höflich (verbeugt)“, wird sein Gruß nicht erwidert. (S.16). Zu Beginn ist sie frech (S.16,Zitat 1), manipulativ (S.17, Zitat 2, S.19 Zitat 3), blauäugig (S.19, Zitat 4) ignorant und hochnäsig (S.23, Zitat 5). Vor ihrer Gefangennahme „sehnte sich (Mohnschnecke), je älter sie wurde, zunehmend nach aufregenden Abenteuern“ (S. 15), sodass sie trotz ihres heftigen Widerstands bei der Gefangennahme durch die Piroggenpiraten (S. 32) mit der Zeit immer mehr Gefallen am Leben auf der Speckkugel findet. Sie ist mutig beim Angriff auf ein venezianisches Handelsschiff (S. 52),  begeistert von der Ausbeute, die sie bei den Raubzügen machen (S. 59 ff.), trägt nun Hemd und Hose (S.80) und zeigt sich ehrgeizig beim Fechtunterricht (S.127, Zitat 6). Zwar führt Käptn Mordan wegen ihr einige Neuerungen in Sachen Hygiene (S. 144 ff) und Morgengymnastik (S.180 f) ein, die der Mannschaft  zusammen mit der Veränderung Morgans sehr eigenartig vorkommen (S.94, S. 129). Mit Hilfe ihrer Ideen, ihrem Mut und ihrer schnellen Auffassungsgabe  hilft sie den Piroggen aber auch mehrere Male aus brenzligen Situationen z.B. als Mordan angegriffen wird (S.100) und als sie die Spunier (S. 254 ff) austrickst. Dafür erntet sie Anerkennung durch die Mannschaft (S. 260, Zitat 7) und befindet sich vorerst auf einem emotionalen Höhepunkt.  Nach  Käptn Morgans Heiratsantrag (S. 297 ff), den sie ablehnt (S.302), woraufhin er beschließt in ein Kloster zu gehen, beginnt sie jedoch an ihrem Leben auf dem Piroggenschiff zu zweifeln (S.302, Zitat 8) und betrachtet ihr früheres Ich kritisch (S.379, Zitat 9). Ihre Unsicherheit verbirgt sie und  wandelt sie stets in einen anderen Gemütszustand um: als Käptn Morgan das Schiff verlässt wird sie wütend (S. 375), als Bonaventura ihr vor der Abstimmung zum neuen Kapitän zur Flucht verhelfen will, weil es gefährlich für sie werden könnte, reagiert sie stolz und hartnäckig (S.378 f.). Auch Eclair fällt die Veränderung auf, als sie sich nach der Schlacht von Trüffelgar auf der Speckkugel treffen (S.461). Zwar meint Mohnschnecke dabei traurig„(d)iese Mohnschnecke gibt es nicht mehr“(S.463) und bekennt sich dadurch zu ihrem Wandel, will diesen aber noch nicht wahrhaben und schickt Eclair und Otto weg (S. 463 f). Als Halbpirogg sich als schlechter Koch erweist (S. 218 ff), gesteht sie sich zwar ein, dass sie ihr altes Leben vermisst, schüttelt den Gedanken zunächst ab (S.521 f), fragt die Mannschaft, wohl auch weil sie sich ihrer Verantwortung als Kapitän bewusst ist, aber letztendlich ob sie nach Sankt Krokant fahren könnten (S.529).Auf Napoleons Ball wird sie festgenommen (S.551). Auch wenn sie bei der Verhaftung gehäßig reagiert (S.552, Zitat 10) und bei der Gerichtsverhandlung „stolz und unbeugsam“ (S.573) auftritt, ist sie dennoch hoffnungslos und befindet sich im Gefängnis auf ihrem emotionalen Tiefpunkt.(S.561) Als ihr Vater ihr anbietet an ihrer Stelle im Gefängnis zu bleiben lehnt sie ab (S.562). Bei der Befreiung fällt sie in Ohnmacht – „zu viel hatte sie in letzter Zeit durchmachen müssen“(S.617). Als sie wieder aufwacht ist sie liebevoll zu Eclair und „(streicht) leicht über (seinen) Kopf“, später erklärt sie ihm den Nachthimmel (S.631) „und küsst(e) (Käptn Morgan) auf die Wange.“


Zitat 1:  „Bah, abwiegen – also nein, was denken Sie sich denn? (…) Mohnschnecke nahm dem Verkäufer kokett das Schäufelchen aus der Hand (…)“

Zitat 2: „Dass sie Eclair gefiel war für Mohnschnecke kein Geheimnis.“
Zitat 3 „Also, dann bis morgen Mittag (…)! rief Mohnschnecke ihm zu, (…) und Eclair blieb nichts anderes übrig, als zu nicken“
Zitat 4: „ (…) dem Mädchen (kam) einer von diesen unschuldigen Einfällen in den Sinn, die manchmal völlig unvorhersehbare und weitreichende Folgen zu haben geruhen“
Zitat 5: „Sein Rücken tat weh, die Hände waren wund, und die Finger zitterten wie die eines alten Zwiebacks. Aber Mohnschnecke bemerkte nichts von alledem. Sie Rüpel hätten mir ruhig mal die Hand reichen können (...), rief sie während sie geschickt auf die Planken der Anlegestelle sprang“
Zitat 6: „Anfangs hatte sie es schwer (…) und dem Mädchen (taten) abends die Arme unerträglich weh. Aber mit der Zeit verwandelten sich die Schwielen an den Handflächen in harte Hornhaut, (und) die Arme wurden immer geübter (...)“
Zitat 7: „Du bist eine echte echte Pirogge!“, sprach Mario stellvertretend den Gedanken aller aus (...)“
Zitat 8: „Das Leben kam ihr mit einem Mal fürchterlich kompliziert vor – alles war so schön gewesen, aber jetzt...“
Zitat 9; „Ihr Zuhause, der friedliche Gewürzladen, die Romane, die sie früher gelesen hatte, der hilfsbereite Verkäufer Eclair – das alles kam ihr so weit weg vor, als hätte irgendein anderes, naives und romantisches Mädchen in einem vollkommen anderen leben das alles erlebt...“
Zitat 10 : „Wie hat dir damals dein Spaziergang über die Planke gefallen?“


\subsubsection{Eclair}

Eclair ist ein höflicher, (S. 16, S. 42 ff) „sehr gut aussehender Jüngling“ (S.15), der zwar gebildet, aber eher schwach und arm ist (S.15, Zitat 1) und aufgrund seiner finanziellen Lage sein Heimatland verlassen musste.(S.15f) Er arbeitet für Mohnschneckes Vater und verheimlicht ihr bei der Frage ob er rudern könne (S.17) seinen Misserfolg, da sie ihm gefällt (S.17). Als er beobachtet, wie Hörnchens Yacht von Piroggen geentert wird, denkt er nur an Mohnschnecke und will sie unbedingt retten (S.29) auch wenn „(er) keine Ahnung (hatte), was er tun sollte, wenn er das Piroggenschiff erreichen würde und wie er Mohnschnecke helfen könnte.“ Da er es nicht schafft sie zu befreien und weil er Hörnchen für die Entführung und den möglichen Tod Mohnschneckes verantwortlich macht, reagiert er , wohl aus Hilflosigkeit, aber auch aus Ärger über Hörnchen, zornig. (S.36) Er wird in das Dorf aufgenommen (S.67, Zitat 3), auf das er nach seiner Nacht am Strand trifft (S.43) und zeigt sich hilfsbereit (S.44, Zitat 2), sozial (S.106ff, Zitat 4) und ist bescheiden (S.67, Zitat 5) Als er mit Otto in See sticht, bemerkt er selbst, dass er sich verändert hat, sowohl äußerlich (S.66, Zitat 6), wie auch persönlich.(S. 137, Zitat 7) Beim lenken der Brigantine, zusammen mit Otto zeigt er sich teamfähig. (S.134) Er ist schlau, was sich bei seinem Angriff auf ein Grützschiff zusammen mit den Reispiroggen zeigt. (S.309 ff) Beim Gedanken an eine Begegnung mit der Speckkugel ist er  unsicher und ängstlich.(S.162) Nach einem Sturm, als Eclair mit Otto im Meer treibt, schließt er aus Verzweiflung und Perspektivlosigkeit über seine Lage schon mit seinem Leben ab. (S.209) Erst als Nachdem er noch ungläubig endlich auf der Speckkugel gelandet ist (S.460), wird er von Mohnschnecke nach Hause geschickt (S.464) und erlebt  nach einem ersten Tiefpunkt, bei dem er angesichts seines Vorhabens unsicher ist (S.414) nach dem Abschied von Otto die Einsamkeit und Traurigkeit (S.117, Zitat 8) zurück im Pelmenidorf . (S. 515f) Mit Mohnschnecke hat er abgeschlossen und aus dieser Lage löst er sich erst, als ihn Otto um Hilfe bei der Befreiung von Mohnschnecke bittet und ihm sagt „Ja, es ist wichtig für mich“ (S.556) Auch wenn ein erster Befreiungsversuch scheitert, gibt Eclair die Hoffnung nicht auf und es offfenbart sich, dass er in Mohnschnecke verliebt ist. (S.603)

(1)“Längere Studien an der Bonbonne hatten seine geistigen Fähigkeiten vervollkommnet, was man leider weder von seinen Muskeln noch von seiner finanziellen Lage sagen konnte.“
(2) „Ich kann arbeiten.“
(3) „Die übrigen Einwohner hatten sich schnell an den ungewöhnlichen Nachbarn gewöhnt.“
(4) „Was hältst du davon, wenn wir die Chance hätten, an einen Satz erstklassiger Segel zu kommen (…) ?“
(5) „Am Rande der Pelmenisiedlung baute Eclair sich im Laufe der Zeit eine kleine Hütte aus Treibgut (…).“ 
(6)“Vom täglichen Rudern wurde die Haut seiner Handflächen allmählich dick und hart.“
(7)“Bei seinen Überlegungen stellte Eclair erstaunt fest, dass der feine Bonbonne – Absolvent und spätere Verkäufer im Mohn\'schen Laden ein völlig anderer zu sein schien, der vermeintlich rein gar nichts gemein hatte mit dem heutigen Eclair – dem Elritzenfischer aus dem Pelmenidorf und Schiffsbaumeister mit den schwieligen Händen und kräftigen Muskeln.“
(8)“Der Himmel war immer häufiger bewölkt, und manchmal wüteten auf dem Meer richtige Stürme. Dann blieb Eclair an der Küste und starrte stundenland auf die Wellen.“


\subsubsection{Hörnchen}

Hörnchen „(sieht) umwerfend aus“ (S.25) und„hat(te) eine geschmeidige, kräftige Figur“ (S.25). Er ist charmant (S.25, Zitat 1), aber hochnäsig und egoistisch: er erzählt sofort von seinem bekannten Vater (S. 25), denkt Eclair sei Mohnschneckes Diener (S.26) und denkt mit Geld kann er sich von den Piroggen freikaufen. (S.34, Zitat 2) Obwohl er sich auf der Speckkugel nur um seine eigene Befreiung bemüht hat, behauptet er heldenhaft gekämpft zu haben. (S.40)  Er ist faul und drückt sich vor Verantwortung, er täuscht vor immernoch verletzt zu sein (S.62 f) und begibt sich erst auf die Suche nach Mohnschnecke, als ihm sein Vater Anweisungen erteilt (S.82 ff). Beim Überschreiten des Grenzübergangs nach Käsien geht sein persönlicher Stolz, über das Allgemeinwohl und er reagiert missmutig auf den Knofiältesten. (S.142 f)
In einer Pizzeria ist er sich zu fein im Rahmen des Reisebudgets zu essen,(S.170f, Zitat 3) widersetzt sich Zwieback und denk respektlos über ihn (S170), er lässt sich ungern Ratschläge geben (S.170, S.219) Als ihm die Pizza Knallpuffer anbietet und er wie selbverständlich bestellt, ist er zu stolz zuzugeben, dass er diese nicht kennt und lässt sich von den Gästen und der Bedienung bei seiner Bestellung schmeicheln und leicht beeinflussen. (S.171) Später gibt er trotz seines eigenen Fehlverhaltens Zwieback die Schuld. (S219) Dass er trotz seiner stolzen Art wenig Selbstbewusstsein hat zeigt sich, als die Käsewurst nach der Schlacht mit den Blutwürsten missverständlicherweise annimmt, dass Hörnchen sie zur Frau nehmen will (S.230), statt das Missverständnis aufzuklären flüchtet er mit Zwieback durch ein Fenster (S.233 ff). Aber hier zeigt sich zum ersten Mal Mitgefühl, er ist Verzweifelt und „kann (…) dem armen Mädchen doch nicht mehr sagen, dass (er) sie nicht mehr heiraten will...“ (S.239) Als Hörnchen und Zwieback in Wursterreich als vermeintliche Spione enttarnt werden und auf dem Weg ins Gefängnis „mit ranzigen Speckstücken (beworfen)“ (S.294) ist er „am Boden zerstört.“ (S.294) Im Gefängnis verliert er zunächst seinen Stolz und isst Gerstengrützbrei  dann lässt sein Widerstand nach, als er Zwieback schimmeln sieht. (S.322) Dass Hörnchen eine eher langsame Auffassgabe hat, zeigt sich nach dem Vorkommnis an der Grenze zu Käsien (S.142), nun auch nachdem er und Zwieback freigesprochen werden: Zwieback versucht aus der Situation Gewinn zu schlagen und Hörnchen fragt „Hatten wir wirklich vierhundert Taler?“ (S.331) Seinen Stolz hat er auch in der Gefängniszelle verloren, wie sich bei der Gerichtsverhandlung (S.336, Zitat 4)und später auf dem Schiff der Paprikapiroggen zeigt. (S. 367, Zitat 5), aber sein Mitgefühl ist gewachsen, er verspricht Alalie, ihren Vater nicht zu verletzen (S.341) und freut sich beim Turnier über die Anwesenheit von Leuten die er kennt.(S.349) Dass eigentlich die falsche Mohnschnecke in Djadida gekauft wurde, ist Hörnchen eigentlich bewusst, aber er scheint das schnell zu vergessen und nach Hause fahren zu wollen. (S.390) Als sich dann herausstellt, dass es sich um eine Zimtschnecke handelt und Hörnchen zum Gespött der Stadt wird, bemitleidet er sich ebenso wie am Anfang selber. (S.514) zieht sich zurück und bereut Alalie nicht geheiratet zu haben. (S.538) Als ihm Halbpirogg seinen Racheplan erzählt, sieht er eine Möglichkeit wieder an im Ansehen der Stadtbewohner zu steigen. (S.540) Allerdings: „Warum fühlte er sich dann so schlecht?“ (S.559)


(1)“Als er Mohnschneckes Interesse bemerkte, zwinkerte er ihr zu und legte zum Gruß zwei Finger an die Mütze.“
(2) „Lasst mich laufen! Mein Vater ist reich, er wird bezahlen!...“
(3)“Ich mache was ich will! Der alte Knochen muss so oder so für mich bezahlen“
(4) „Dann wird der Weißbrotbursche..., begann der Richter, doch Hörnchen unterbrach ihn . `Hörnchen, mit Verlaub. Mein Name ist Hörnchen,“
(5) „Glaubst du wirklich, ich würde...,“

\subsection{Wichtige Nebenfiguren}
\subsubsection{Mordan}
Cristóbal Mordan ist der Kapitän der Speckkugel (S. 636), „(hat) ein Holzbein (und trägt) eine schwarze Augenklappe, einen mit befransten goldenen Epauletten verzierten schwarzen Uniformrock“ (S.31) und einen  Krummsäbel und zwei silberne Pistolen bei sich. (S.31) Er kümmert sich von Anfang an um Mohnschnecke und zweifelt nie an ihrer Anwesenheit auf dem Schiff. Zwar werden Gefangene üblicherweise auf dem Markt in Djadida verkauft (S.50), aber Mordan denkt nicht einmal daran sie zu verkaufen, im Gegenteil:  er lässt sie zu ihrem Schutz vom Mast losbinden, als die Speckkugel in ein Gefecht gerät (S.52f.), überlässt ihr die Kapitänskajüte und schläft in einer Hängematte (S.77 ff) und fängt an sich zu waschen (S. 129). Er zeigt ihr gegenüber immer mehr Vertrauen (S.102, Zitat 1) und es zeigen sich erste Anzeichen von Sympathie und Verliebtheit: „Nur wenn Mohnschnecke auf Deck erschien (…), unterbrach Mordan sein Gefluche“ (S.94), er wird nervös wenn er sich verspricht (S.102), macht ihr Geschenke (S.102), bewahrt drei Mohnkörnchen von ihr auf (S. 182) und weicht Fragen zu ihrem Verkauf entweder aus oder reagiert gereizt (S.129, S.145). Die Neuerungen, die er wegen ihr einführt sieht er positiv, „denn eine saubere Mannschaft sah viel fescher aus und litt auch seltener an Schimmelschnupfen oder verdorbener Füllung“ (S.181) Als Halbpirogg versucht die Mannschaft gegen sie aufzubringen, geht er nicht weiter auf die Anschuldigungen ein und macht seinen Standpunkt klar, indem er Halbpirogg schlägt. (S.188) Er verbringt viel Zeit mit Mohnschnecke, erklärt ihr die Sterne (S.207) zeigt ihr den Umgang mit dem Sextanten (S.297) Als er ihr einen Heiratsantrag macht,ist er nervös und versucht ihr deswegen zunächst Komplimente zu machen, doch auch als er sich überwinden kann,  lehnt sie ab (S.300) und er ist am Boden zerstört, sodass er seine Kajüte vier Tage lang nicht verlässt. (S.302) Verletzt und weil er einsieht, dass Mohnschnecke der tapferere, bessere Kapitän (S.300) und jünger ist als er (S.371) verlässt er das Schiff und überträgt ihr die Verantwortung. Er flieht in die Einsamkeit eines Klosters (S.400 ff) und kann Mohnschnecke dennoch nicht vergessen (S.403, Zitat 2) Er hat Schwierigkeiten sich ans Klosterleben zu gewöhnen, zeigt sich uneinsichtig und kann seine Piroggeneigenschaften nicht ablegen. (S.467) Sein Widerstand schwindet jedoch als er eingemauert wird hoffnungslos, traurig und reumütig (S.582ff): „der Anblick der winzigen schwarzen Körnchen erschütterte ihn so sehr, dass Mordan das kleine Behältnis wieder in der Tasche verstaute und nicht wieder hervorholte.“ (S.531) Als er jedoch von ihrer Hinrichtung erfährt , überwindet er alle Widerstände und rettet sie (S.620) Zwar hegt er immernoch eine gewisse Zuneigung zu ihr, akzeptiert aber, dass sie mit Eclair zusammen ist. (S.632f)

„Verlegen überreichte er ihr das Kästchen (mit Duellpistolen) und wandte sich ab.“
„In der Dose lagen drei winzige Mohnkörnchen“

\subsubsection{Halbpirogg}
Halbpirogg ist der Kapitänsmaat. Seinen Namen hat er seit ihm ein Gegner bei einer Schlacht 
einen beträchtlichen Happen aus seiner Seite gesäbelt hatte (...)(dort) 
war ihm ein Teil der Füllung herausgepurzelt (...) 
(und) (dort wurde er) mit Hanfgarn zugenäht\cite[S. 33]{pir}
Er bemerkt als Erster die Veränderungen des Kapitäns nach 
dem Auftauchen Mohnschneckes, stichelt und ist ihr gegenüber pessimistisch:
er redet ihre Vorschläge schlecht \cite[S. 266]{pir} schimpft immer wieder über Diese verflixte Mädel\cite[S. 145]{pir}
die verflixten Weiber \cite[302]{pir}und 
hegt ihr gegenüber eine negative Grundeinstellung. \cite[S. 297]{pir}
Als er Käptn Morgan daran erinnert, dass Mohnschnecke, wie es eigentlich üblich ist (S.50)) 
auf dem Markt verkauft werden soll (S.129), stößt dabei immer wieder auf wenig Aufmerksamkeit. \cite[S. 145]{pir}
Seine schlechte Laune ihr gegenüber, ist in Käptn Morgans Aufmerksamkeit gegenüber Mohnschnekce begründet: er macht
sich Hoffnungen irgendwann Käptn Morgans Nachfolge anzutreten, 
da er (a)ls ältestes Besatzungsmitglied (...) die meiste Erfahrung (...) (hatte)\cite[S.371]{pir}
 
Da er (e)ine Menge Mannschaftsmitglieder  (...) im Lauf der Zeit gegen sich aufgebracht (hatte), \cite(S.151), 
stürzen die sich beim Waschen mit Vergnügen auf ihn (S.151)
Aus Frust über die mangelnde Anerkennung und die Veränderungen an Bord, 
(stattet) er in letzter Zeit immer häufiger den im Laderaum gestapelten Rumfässern 
einen Besuch (ab) \cite[S. 181]{pir} und sucht diese immer wieder auf, wenn er sich ärgert \cite[S. 303]{pir}


\subsubsection{Pelmen Otto}


\subsection {Personenkonstellationen}
\subsubsection {Pelmeni}
\subsubsection{Speckkugelmannschaft}
Der anfängliche Unmut der Mannschaft gegenüber Mohnschnecke und den Veränderungen, die mit ihr auf der 
Speckkugel einkehren \cite[S. 77 f]{pir} verfliegt, als die Piroggen Gefallen an 
\subsection{Schauplätze}
\subsubsection{Ursprungsorte}
Mohnschnecke stammt aus Murseille und träumt seit ihrer Kindheit dort davon auf ein Gartenfest von Graf Napoleon zu gehen.

Eclair stammt aus einem verarmten Adelsgeschlecht und hat an der Bonbonne studiert.

\subsubsection{Verweilorte}

Murseille, Speckkugel, Pelmenidorf 

Wichtige Handlungsorte
		
\chapter{Strukturanalyse}

\section{Handlung}
\subsection{Aufbau}
„Die wilden Piroggenpiraten“ ist unterteilt in einen Prolog, 67 Kapitel und den Epilog. Im Prolog wird der Leser in die Geschichte und den Kosmos der Piroggen eingeführt. Die Handlung lässt sich in vier Teile unterteilen, die durch drei Ereignisse abgetrennt werden: das Übergangskapitel 4, nach dem Eclair und Hörnchen getrennte Erzählstränge – also auch wie Mohnschnecke eigene Kapitel haben – die Schlacht von Trüffelgar (Kapitel 41 – 44) und Mohnschneckes Gefangennahme (Kapitel 55- 65) . Der erste Teil (Kapitel 1-3) handelt von Mohnschnecke, der Hauptprotagonistin, Eclair und Hörnchen. Am Ende des dritten Kapitels wird Mohnschnecke entführt. Im zweiten Teil (Kapitel 5-42) machen sich Eclair, begleitet von Otto (Kapitel 13), und Hörnchen (Kapitel 14) auf den Weg Mohnschnecke zu suchen. Bis auf einen kurzen Berührpunkt der Erzählstränge in Kapitel 22, gehen alle drei Figuren getrennte Wege und haben eigene Kapitel. Vor und bei der Schlacht von Trüffelgar sind Eclair und Mohnschnecke in denselben Kapiteln zu treffen, da sie sich in unmittelbarer Nähe befinden und sich schließlich auf er Speckkugel treffen,  Mohnschnecke schickt Eclair und Otto aber zurück nach Murseille, sodass die Protagonisten wieder in abgegrenzten kapiteln zu finden sind. Hörnchen bringt währenddessen eine verschleierte Zimtschnecke nach Murseille, weil er sie für Mohnschnecke gehalten hat. Im dritten Teil entschließt sich Mohnschnecke auf Graf Napoleons Ball zu fahren, wo sie verhaftet wird. Die Erzählstränge überkreuzen sich hier häufiger, in Kapitel 53 kommt Mordan im selben Kapitel vor wie Mohnschnecke und Halbpirogg, obwohl er im Kloster ist, im Darauffolgenden schmieden Hörnchen und Halbpirogg zusammen einen Plan, den sie in Kapitel 55 ausführen und Mohnschnecke gefangen nehmen lassen. In Kapitel 59 sehen sich das Mädchen und Eclair im Gerichtssaal. 

\subsection{Kernhandlung}
In der Geschichte dreht sich alles um das Mädchen Mohnschnecke. Sie wohnt in Murseille und möchte auf einen Ball nach Sankt Krokant, wo sie bei einem Yachtausflug von Piroggen entführt wird. (Kapitel 1-3) Mordan, der Kapitän verliebt sich in sie und führt wegen ihr einige Neuerungen ein, die der Mannschaft zunächst nicht gefallen. (Kapitel 15) Durch ihr selbstbewusstes Auftreten, ihren Ehrgeiz in Kampfangelegenheiten und ihre Intelligenz, mit der sie den Seeräubern mehrere Male hilft, lebt sie sich schnell auf der Speckkugel ein. Als ihr der Kapitän einen Heiratsantrag macht lehnt sie dankend ab, woraufhin Morgan das Schiff verlässt und Mohnschnecke beginnt nun ihr eigenes Denken und Handeln zu hinterfragen. Bei der Schlacht von Trüffelgar bei der der Großteil der weltweiten Piroggen gegen die Spunier kämpft, setzt sich das Mädchen als Anführerin durch. Eclair, dem sie danach begegnet und der sie gern nach Hause bringen möchte, weist sie ab und schickt ihn weg. Dennoch lässt sie der Gedanke an den Ball auf Sankt Krokant nicht los, sodass sie mit der Speckkugel bis kurz nach Sankt Krokant segelt. Verkleidet auf dem Ball um nicht erkannt zu werden, wird sie festgenommen: Halbpirogg und Hörnchen haben sie verraten. Eclair und die Speckpiroggen versuchen mit allen Mitteln sie zu befreien.

\section{Rahmenhandlung}

Die Rahmenhandlung bilden Eclair und Hörnchen.

Zunächst rudert Eclair mit Hörnchen nach Mohnschneckes Entführung zurück nach Hause. Im Hafen, wo Hörnchen von seinem angeblichen Kampf berichtet, kann der geschockte Eclair Mohnschneckes Vater kaum widersprechen und stößt, nachdem er in Mohnschen Haus nicht mehr willkommen ist und kein Geld mehr hat, nach einer Nacht am Strand auf eine Pelmenisiedlung. Dort lernt er Otto kennen, mit dem er eine Brigantine repariert, um mit ihn zu einem Piroggenbackofen zu begleiten und nach Mohnschnecke zu suchen. Als die beiden nach einem Sturm schiffbrüchig sind, werden sie von Reispiroggen gerettet, mit denen sie zur Schlacht von Trüffelgar fahren. Nachdem Mohnschnecke sie dort nach einem Sieg entlohnt und nach Hause schickt, begleitet Otto seinen Freund zunächst in einen Piroggentempel und, nachdem das Backen dort aufgrund aufgebrachter Piroggen scheitert, auf einen Vulkan, wo Otto der Pelmen sich schließlich seinen Traum erfüllt. Zunächst gehen die Beiden nun getrennte Wege, Otto schließt sich den Piroggen auf der Speckkugel an und Eclair kehrt zurück ins Pelmenidorf, aber als Mohnschnecke gefangen genommen wird will Eclair seinem Freund helfen sie zu befreien.

Hörnchen liegt zunächst faul zu Hause und bricht erst zu einem Befreiungsversuch auf, als ihn sein Vater losschickt. Er schließt sich zusammen mit Zwieback, seinem Begleiter, einer Knofikarawane an. Nach einer Schlacht mit Blutwürsten denkt die gerettete Käsewurst Alalie, dass Hörnchen sie zur Frau nehmen will, sodass ihr Vater bei ihrer Ankunft eine Verlobungsfeier vorbereitet. Da dies nie seine Absicht war, flieht Hörnchen mit Zwieback und wird von Alalie verabschiedet, die ihnen noch Blutwursthäute als Verkleidung mitgibt. Mit diesen überschreiten sie die Grenze nach Wursterreich, wo sie als Spione auffliegen und verhaftet werden. Bei ihrer Verhandlung werden sie mit Hilfe des Knofiältesten zwar freigesprochen, der Einspruch des Vicomte de Brinson führt jedoch dazu, dass sich er und der Vicomte duellieren müssen. Nach einem Sieg fahren Hörnchen und Zwieback auf der Straße von Djadida auf den Sklavenmarkt und kaufen dort ein vermummtes Mädchen, das sie für Mohnschnecke halten. Zurück zu Hause stellt sich herraus, dass es sich um eine Zimtschnecke handelt. Da Hörnchen nun das Gespött der Stadt ist, ist er umso empfänglicher für Halbpiroggs Racheplan Mohnschnecke beim Frühlingsball verhaften zu lassen.


\section{Spannungskurve}


\subsection{Höhepunkte}
Das Buch enthält viele verschiedene Spannungskurven und die sich teilweise auf einzelne Personen aber auch Personengruppen beziehen.  Eine Spannungskurve erstreckt sich vom Kapitel eins bis drei, bis zur Entführung Mohnschneckes. Die erste Auseinandersetzung mit den Spuniern (Kapitel 26) schauckelt sich über Kapitel 27 und 28 hoch, flacht dann wieder ab, bis im Kapitel 39 die Nachricht von den Pilzpiroggen über die Schlacht von Trüffelgar überbracht wird, die dann der Höhepunkt ist. Eine Kurve, die sich langsam aufbaut ist die von Halbpirogg: ab Kapitel 18 kommt immer wieder sein Unmut über Mohnschnecke zum Ausdruck, der von seiner Niederlage im Kampf um den Kapitänsposten und die Ablehnung durch die Mannschaft wegen seines schlechten Essens,  genährt wird. Er findet seinen Höhepunkt in seiner Rache beim Verrat auf dem Frühlingsball und dem Versuch der Übernahme der Speckkugel . Ab Mohnschneckes Festnahme steigt auch hier eine Spannungskurve steil an, die Ereignisse überschlagen sich und der Erzählstrang wird immer mehr wieder zu einem zusammengefügt, die Ansammlung immer mehr, vorher unabhängig erzählter Erzählperspektiven häufen sich. Kleinere Spannungskurven erstrecken sich über einzelne Kapitel. Zum Beispiel Mohnschneckes erste Schlacht, der Angriff auf Mordan, den sie abwehrt, Mordans Heiratsantrag und als sie verkleidet auf den Frühlingsball geht ; Eclairs List, mit der er und Otto zum letzten fehlenden Schiffzubehör kommen, das Kapitel in dem sie vom Reispiroggenschiff gerettet werden und als Otto zunächst versucht im Piroggentempel und dann auf dem Vulkan gebacken zu werden. Hörnchens zunächst eher unspektakulärer Weg, wird mit der Blutwurstschlacht spannender. Die darauffolgende vermeintliche Verlobung, die zuerst in Vergessenheit gerät führt nach der Festnahme in Wursterreich, bei der Gerichtsverhandlung zu einem Höhepunkt: dem Turnier. Nachdem er dieses als Sieger verlässt und eine vermummtes Mädchen kauft, wird es bis zur Enthüllung in Murseille spannend.

\subsection{Wendepunkte}

Die Entführung Mohnschneckes setzt das von Eclair und Hörnchen angestrebte Ziel der Geschichte: ihre Befreiung.  Die Veränderungen an Bord, die durch sie indiziert werden, führen zu einem anderen Verhalten bei Halbpirogg, (Kapitel 18) der durch seinen Unmut die Geschichte mit seinem Verrat später entscheidend beeinflusst. Mordans Heiratsantrag, den Mohnschnecke ablehnt, bildet die Grundlage für ihre Beförderung zum Käptn und einer Veränderung ihrer Stimmung. 

Zunächst scheint es als hätte Eclair aufgegeben, doch dann erzählt ihm Otto im Pelmenidorf von der Brigantine und von seinem Wunsch ein Pirogg zu werden. Als die Beiden nach einem Sturm hoffnungslos im Meer treiben, werden von Reispiroggen auf ihrem Schiff aufgenommen und fahren mit zur Schlacht von Trüffelgar. Eigentlich scheint Eclairs Ziel erreicht, als er Mohnschnecke nach der Schlacht auf der Speckkugel begegnet, doch für ihn unerwartet, will das Mädchen nicht mit nach Hause kommen. Er geht zurück ins Pelmenidorf, wo er später von Otto, der, nachdem er mit Eclair auf einem Vulkan war, ein Pirogg ist um Hilfe bei der Befreiung Mohnschneckes.

Erst wegen seines Vaters fängt Hörnchen an nach Mohnschnecke zu suchen. Die Verkleidung, die er von Alalie bekommt, ist zwar bei der Grenzüberschreitung nach Wursterreich von Nutzen, aber führt auch zur Festnahme als Spion und bei der Verhandlung zum Turnier. In Djadida glaubt er die vermummte Mohschnecke gefunden zu haben, zu Hause stellt sich herraus, dass er hinters Licht geführt wurde. 
Entscheidend für das Happy End ist auch, dass die Nachricht über Mohnschneckes Verhaftung Mordan im Kloster erreicht, sodass er sie überraschend am Tag ihrer Hinrichtung retten kann.


Erzähltechnik
	
Erzählperspektive

Die Geschichte wird sowohl auktorial, als auch personal widergegeben. Der Erzähler hat einen Überblick über das Gesamtgeschehen und berichtet  beobachtend sowohl über Vergangenheit, als auch Gegenwart, was sowohl im Prolog, im Epilog, als auch in der Geschichte selbst geschieht. 
Im Prolog wird der Leser zunächst in die Welt der Piroggen eingeführt und lernt die Stadt Murseille kennen. Während des Handlungsverlaufs, bei denen die Erzählstränge der einzelnen Protagonisten parallel ablaufen, kann der Erzähler sowohl aus der Sicht einer oder mehrerer Personen vom aktuellen Geschehen berichten, als auch auf vergangenen Ereignisse zurückgreifen und vermittelt dadurch die Welt der Wirklichkeit der Protagonisten. Zum Beispiel kennt er die Vergangenheit und den Hintergrund Graf Napoleons (S.17f), erklärt die Eigenschaften der Bewohner von Käsien (S.140f) und kennt die Intentionen, Gedanken und Gefühle Mohnschneckes als Mordan das Schiff verlässt (S.374). Als Mohnschnecke entführt wird (Kapitel 3) die Speckkugel an Eclair und Otto vorbeifährt (Kapitel 22) und bei der Gerichtsverhandlung (Kapitel 59) wird gleichzeitig aus mehreren Perspektiven erzählt.



Erzähltechnik

Es sind verschiedene Erzählweisen zu finden. Meist wird berichtend erzählt, der Erzähler beschränkt sich auf Beschreibungen und berichtet dabei  subjektiv, etwa über die Geschehnisse an Bord und als sich „Halbpirogg (…) sich wieder einen hinter die Binde gegossen (hatte)“ (S.298) und bei der Schlacht von Trüffelgar, „als in der Meerenge (…) wahrlich merkwürdige Dinge (geschahen)“ (S.444). Szenisches Erzählen wird verwendet, als im Hafen auf die Brigantine und die Einlösung der Wettschuld gewartet wird (S.117), der Dialog zwischen Hörnchen und Alalie vor dem Turnier (S.340) und die Unterhaltung zwischen Eclair und Otto vor der Schlacht bei Trüffelgar (S.414ff) Als Eclair und Otto schiffbrüchig sind, werden seine Gefühle in Erlebter Rede zum Ausdruck gebracht (S.209f, Zitat 1), ebenso als Mordan das Schiff verlässt und Mohnschnecke die Verantwortung überlässt,(S.375, Zitat 2).


„Er fühlte sich völlig aufgeweicht.“ „Er wollte gar nichts. In seinem Körper hauste keinerlei Verlangen mehr, höchstens der Wunsch, einzuschlafen und nie mehr aufzuwachen.“
„Verfluchter Mordan“ Befördert sie auf den Kapitänsposten ohne sie überhaupt zu fragen, ob sie das möchte!“

Erzählzeit

Im Laufe der Handlung kommt es sowohl zu Zeitraffungen, als auch zu Zeitdeckungen in der Erzählzeit. In den einzelnen Kapiteln kommt es generell zu einer Zeitdeckung, die erzählte Zeit und die Erzählzeit sind in etwa gleich lang. Bei der Entführung Mohnschneckes in Kapitel 3 und bei „Admiral Morcillas letztes Gefecht“ (kapitel 44.) wird zeitgleich aus mehreren Perspektiven erzählt. Ansonsten kommt es meist zu Zeitraffungen, da die Erlebnisse der einzelnen Hauptprotagonisten abwechselnd erzählt werden.  Nach dem Berührpunkt mit Eclair in Kapitel 22, erzählen bis zu Schlacht von Trüffelgar 6 Kapitel von Mohnschnecke und lediglich 3 von Eclair.

Sprachanalyse
 
Semantik


In der Linguistik wird mit Hyperonym (von griechisch hyper ‚über’ und onyma ‚Name’)[1] der Oberbegriff eines Begriffs bezeichnet. Der Unterbegriff eines Begriffs wird als Hyponym bezeichnet.
Beispiel: Obst ist ein Hyperonym von Banane, Tier ist ein Hyperonym von Säugetier, Säugetier ist ein Hyperonym von Hund.
Ein bestimmtes Wort kann je nach Blickrichtung beides sein: So ist Hund Hyponym in Bezug auf Säugetier und Hyperonym in Bezug auf Basset.


Lebensmittel
Nautik

Stilistik

Aliteration Titel
Erzählstil
Anspielungen
					
					


Einsatz im Unterricht
Pädagogischer Wert
Das Kinderbuch „Die wilden Piroggenpiraten“ ist für Kinder ab 8 Jahren geeignet und bietet vielfältige Möglichkeiten im Unterricht. Durch die vielen einzelnen Kapitel fällt es leicht, zur Entspannung nach einer anstrengenden Lernphase oder als Belohnung nach einem gelungenen Tag, ein oder mehrere Kapitel vorzulesen, ohne dabei mitten in einer Geschichte stoppen zu müssen. Auch die Abgrenzung der einzelnen Charaktere in den verschiedenen Kapiteln ermöglicht nur ausgewählte Sequenzen in den Unterricht einzubauen. Die Lernbereiche im Deutschunterricht können vielfältig gefördert werden. Im Lernbereich „Sprechen“  kann aus der Lektüre vorgelesen, im Unterrichtsgespräch die Handlung zusammengefasst und in Gruppenarbeit im Text vorkommende Fachbegriffe recherchiert werden, die dann von den „Experten“ einer jeden Gruppe erklärt werden. „Sprache untersuchen“ und „Richtig schreiben“ lässt sich anhand der vielen verschiedenen Lebensmittel, die beschrieben werden. Auch unterstützt die Lektüre „Ziele für den Literaturunterricht“: Sie fördert bei richtiger Anwendung im Unterricht und im Klassenzimmer die „Lesefreude“ und beeinflusst bei gelungener Umsetzung die „Texterschließungskompetenz“ Die ungewöhnlichen Protagonisten und die ungewohnte Umgebung, sowie die detaillierte Beschreibung der Szenen und Personen, fördern „Imagination und Kreativität“. Da das Buch wegen der Wahl einer weiblichen Hauptfigur in einer eigentlich typischen Jungengeschichte kein geschlechterspezifisches Publikum anspricht und die Geschichte sich also sowohl um weibliche als auch männliche Protagonisten dreht, bietet sich für jeden Schüler die Möglichkeit sich persönlich mit einer der Figuren zu identifizieren. Auch die vielen verschiedenen Personen tragen zur „Identitätsfindung und (zum) Fremdverstehen“ bei und können bei der „Auseinandersetzung mit anthropologischen Grundfragen“ helfen: Was hat sich die Blutwurst in dieser Situation gedacht und warum hat sie so gehandelt? (S.229)



Beispiele von Verwendung im Unterricht gemäß des Lehrplans

Lehrplan für die bayerische Grundschule 2000
Klasse
Die Schüler erzählen zu Beginn einer Unterrichtseinheit nach, was in der vorrangegangenen Geschichte passiert ist.
Deutsch
Sprechen und Gespräche führen , I
3.11 Einander erzählen und einander zuhören: interessant und spannend erzählen Geschichten nacherzählen

Mohnschnecke schreibt einen Brief an ihre Eltern, in dem sie beschreibt, wie es ihr auf der Speckkugel geht. Mohnschnecke und die Piroggenpiraten sind in einer Lagune gefangen, erzähle, wie sie sich daraus befreien. (Kapitel 26)
4.Klasse
Deutsch
3.2für sich und andere schreiben
Zu Texten schreiben, auf Texte antworten

Lesen und mit Literatur umgehen
3.4.2 lesetechniken weiterentwickeln
Heimat- und Sachunterricht
Zusammenleben   3.4.2 Menschen arbeiten Einen Betrieb /eine Organisation in der Region erkunden   –  handwerksbetrieb


klasse
4.1 Sprechen und Gespräche führen 
4.1.1 Einander erzählen und einander zuhören   
Die Erzählperspektive wechseln

http://www.lehrplanplus.bayern.de/jahrgangsstufenprofil/grundschule/3

3. Unterrichtsbeispiele




Deutsch

HuS

Kunst

Literaturverzeichnis
Anhang


Bilderwelt der Medien
Bewegte Bilder und ihre Helden
Eigene „Helden“ mit besonderen Attributen unter
bewusstem Einsatz von gestalterischen Mitteln
schaffen
Bewegungen durch schnelle Abfolge einzelner Bilder
erzielen, z. B. Daumenkino; mit Videokamera oder
fotografischen Bilderfolgen (Fotoroman)
experimentieren
Eigene „Helden“ mit besonderen Attributen unter
filmisch/fotografisch in Szene setzen:
Sciencefiction-Figur, Helden aus der
Heimatgeschichte
4 Didaktische Analyse

 Ich drehe einen kurzen Ausschnitt aus dem Buch „Die wilden Piroggenpiraten“

Das Unterrichtsthema „Ich drehe eine kurzen Ausschnitt aus dem Buch Die wilden  Piroggenpiraten“ lässt sich im Lehrplan der Jahrgangsstufe 4 unter „Bilderwelt der Medien“ in Anlehnung an „4.4 Bewegte Bilder und ihre Helden“ unter dem besonderen Gesichtspunkt von „Eigene „Helden“ mit besonderren Attributen unter bewusstem Einsatz von gestalterischen Mitteln schaffen“ einordnen. Den Schülergruppen wird eine beschränkte Auswahl an Kapiteln oder Seiten aus dem Buch „Die wilden Piroggenpiraten“ zur Verfügung gestellt. Daraus analysieren sie Ausschlaggebende Merkmale ihrer Hauptfigur und überlegen wie sie diese filmisch/fotografisch in Szene setzen können.

 Didaktische Reduktion

Der erste Teil der Arbeit konzentriert sich auf die Analyse des ausgewählten Textes, mit Hilfe von im Deutschunterricht angeeignetem Vorwissen. Dann wird überlegt wie sich das erarbeitete am Besten darstellen lässt.
 Schülerbezug zum Thema
Im Unterricht ist das Buch „Die wilden Piroggenpiraten“  schon mehrere Male zum einsatz gekommen. 
5 Lernziele

Ziel des Unterrichtsversuches

Die SchülerInnen lernen verschiedene Ausdrucksweisen, wie Mimik und Gestik, sowie Charaktereigenschaften filmisch oder fotografisch darzustellen..


Feinziele

Die SchülerInnen lernen Ideen und Werke Friedensreich Hundertwassers kennen.
Die SchülerInnen erfassen die Eigenschaften und die Individualität des Hundertwasserhauses, auch speziell seiner Fenster.
Die SchülerInnen erkennen Muster, Formen und Farbigkeit und können diese selbst mit Wachsmalkreide, Malkasten und Lackstift umsetzen
Die SchülerInnen gestalten orientiert am Künstler ein eigenes Fenster nach den eigenen Vorstellungen

6 Methodische Analyse

 Medien und Arbeitsmittel

vorbereitete Medien und Arbeitsmittel
Kameras 
Videokameras

benötigte Arbeitsutensilien
gegebenenenfalls Stift und papier


 Begründung der methodischen Planung

Zeitplanung

Da der Unterrichtsversuch durch eine Pause unterbrochen wird und sich nicht im üblichen Rahmen von zwei Schulstunden umsetzen ließ, konnte man die Arbeitsphasen in das Gestalten mit Wachsmalkreide vor und das Gestalten mit dem Malkasten nach der Pause unterteilen. Der letzte Teil, das Arbeiten mit Lackstiften wurde an einem anderen Praktikumstag fertiggestellt.

Medieneinsatz

Die Stunde beginnt mit zwei Folien für den Overheadprojektor. Da das Bild für alle gleichermaßen sichtbar in einem vergrößerten Ausschnitt an die Wand projeziert wird, können es alle SchülerInnen gleichzeitig betrachten und man kann es bei Bedarf noch vergrößern um mehr ins Detail zu gehen. Zuerst ist die Folie mit dem Hundertwasserhaus in Wien zu sehen um wesentliche Unterschiede zu anderen Häusern zu erkennen und die darauf folgenden Fenster in einem größeren, bekannten Zusammenhang einordnen zu können.

Schülerarbeitsmittel

Damit die Schüler die Möglichkeit haben zwei Fenster zu Gestalten, der Ausschnitt aber nicht zu groß werden sollte, damit alle Arbeiten besser zusammen aufgehängt werden können, wurden vor Beginn der Stunde  Blöcke alle Blöcke mit einem vorgezeichneten Rahmen und zwei Fenstern versehen.

Ergebnissicherung

Um die Grundidee des Fensterrechts noch einmal aufzunehmen und die Werke dementsprechend präsentieren zu können werden sie nebeneinander zu einem Rechteck aufgehängt, sodass das Ergebnis wie ein großes Haus aussieht.

Artikulationsschema




Folgestunde



7 Ergebnissicherung
