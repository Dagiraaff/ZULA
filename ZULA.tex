\chapter{Einleitung}
\section{Maris Putnins}
Maris Putnins wurde 1950 in Valmiera, Lettland geboren ist als `Schauspieler, Puppenspieler und Bühnenbildner' 
tätig (vgl.1) und hat sowohl als Drehbuchautor als auch als Produzent bei mehreren Filmen, Kurzfilmen und einer 
Fernsehserie mitgewirkt, die hauptsächlich in Lettland bekannt  sind, (vgl. 2) darunter „Die kleinen Bankräuber“ 
(Originaltitel: „Mazie laupitaji“ – 2009), der von einem kleinen Jungen handelt, der plant, zusammen mit seiner 
Schwester die Bank zu überfallen, die soeben seine Eltern aus der neuen Wohnung verwiesen hat, weil sein Vater 
seinen Arbeitsplatz verloren hat. (siehe 3) 
Sein Buch `Die wilden Piroggenpiraten' wurde 2013 für den Deutschen Literaturpreis nominiert und ist für Kinder 
ab 8 Jahren geeignet.(4)

\section{Formulierung der Fragestellung}
Im Folgenden wird das Buch „Die Piroggenpiraten“, literarisch analysiert. Nach dem ersten Eindruck folgt ein
Einblick in den Handlungsablauf, die Hauptprotagonisten werden charakterisiert und wichtige Schauplätze aufgezeigt. 
Bei der Strukturanalyse wird der Handlungsaufbau dargelegt, Spannungskurven und Erzähltechnik erläutert. 
Bei der Sprachanalyse wird speziell auf die Semantik im Zusammenhang mit Lebensmitteln und Nautik, sowie 
den Erzählstil und (geschichtliche) Anspielungen eingegangen. Nach einer Erklärung wie man das Buch im Unterricht verwenden kann und welche Stellen im Lehrplan für eine Verwendung relevant sein können, schließt die Arbeit mit Unterrichtsbeispielen und einer Erläuterung ob das Buch im Unterricht anwendbar ist. Der Einfachheit halber werden die Figuren auch als „Personen“ bezeichnet oder in einer anderen 
Form menschlich benannt, z.B. Mädchen, Mann, Kind usw. da es sich um personifizierte Lebensmittel handelt, die sich 
ebenso wie Menschen bewegen, denken, fühlen und handeln. Falls nicht anders angegeben wird aus dem 
Buch `Die wilden Piroggenpiraten'(5) von Maris Putnins zitiert.

\section{Erster Eindruck}
Das Titelbild zeigt drei Personen, die an der Spitze eines Schiffes stehen, das von Möwen umkreist wird. Die weibliche Person ist rothaarig und sommersprossig, hält sich mit der Linken an einem Seil fest und hat in der rechten Hand ein Schwert. Ein schlanker junger Mann trägt eine rote Mütze und wird von einem korpulenten Herrn verdeckt, der sowohl eine Kanone als auch ein Messer bei sich hat. Die zwei männlichen Personen stehen hinter dem Mädchen, alle Drei blicken nach rechts, in die Richtung in die sie sich bewegen. Aufgrund der Art ihrer Kleidung, der mitgeführten Accessoires und dem sie umgebenden Meer scheint es sich um Piraten zu handeln. Der Bildaufbau mit der Dynamik nach Rechts erzeugt den Eindruck des Aufbruchs, des Tatendrangs und in Verbindung mit der Mimik der Dargestellten, der Vorfreude. Womöglich sind sie gerade dabei auf einer Insel zu landen um einen Schatz zu suchen oder am Festland eine Stadt zu überfallen, zumindest lassen die Möwen darauf schließen, dass sie sich in Küstennähe befinden.
Der Titel `Die wilden Piroggenpiraten' bestätigt die Annahme, dass es sich um Seeräuber handelt und erzeugt zusammen mit dem Untertitel Neugier, da von einer `entführten Mohnschnecke' die Rede ist und man dieses Gebäck zunächst nicht mit dem Titelbild in Verbindung bringen kann. Erst in Verbindung mit dem Klappentext auf der Rückseite erschließt sich, dass es sich bei der Mohnschnecke, ebenso wie bei `eine(r) wilde(n) Pirogge', `ein(em) Hörnchen' und `ein(em) Eclair' um personifizierte Gebäckstücke handelt. Aufgrund der äußeren Aufmachung, den erwähnten, scheinbar lebendigen Teigwaren, den Schauplätzen - Piratenschiff, Kloster und Kerker- und dem ungewöhnliche Schlachtruf `Macht sie zu Semmelbröseln!' scheint es sich um ein fantastisches, spannendes und wildes Piratenabenteuer zu handeln.

\chapter{Die wilden Piroggenpiraten}

\section{Inhaltsanalyse}
Die Inhaltsanalyse gibt Einblicke in die Handlung, die Charaktere, wichtige Peronenkonstellationen, Schauplätze und schließlich in die Religion.

\subsection{Handlung}
Nach einem kurzen Überblick über die Geschehnisse, folgen die Anfangs- und Endsituation und wichtige Wendepunkte der Geschichte.

\subsubsection{Überblick}
Mohnschnecke fährt mit Eclair, einem Angestellten ihres Vaters, nach Sankt Krokant um zu einem Ball des 
Grafen Napoleon zu gehen. Im Hafen wird sie von Hörnchen auf einen Yachtausflug eingeladen und auf See von 
Piroggen entführt(Kapitel 1-3) 
Während sie sich mit der Zeit auf dem Schiff der Seeräuber einlebt, versuchen Eclair 
und Hörnchen unabhängig voneinander das Mädchen zu retten. Eclair kommt zunächst in ein Pelmenidorf, lernt 
dort Otto kennen und macht sich mit seiner Hilfe auf einer Brigantine auf die Suche nach Mohnschnecke. 
Als die beiden schiffbrüchig und von einem Reispiroggenschiff aufgenommen werden, segeln sie zusammen mit 
ihren Rettern nach Trüffelgar. Mohnschnecke, die Hauptfigur lebt sich auf dem Piroggenschiff ein, wird nach 
Mordans Abschied ins Kloster Kapitän und führt die Piroggen in der Schlacht von Trüffelgar an. (Kapitel 43) 
Aber sie sehnt sich trotz der schönen Zeit bei den Piroggen nach dem Ball, von dem sie schon seit ihrer Kindheit 
träumt. \cite[S.521]{pir} Verraten von Halbpirogg und Hörnchen, die zusammen Rache geplant haben, wird sie wegen ihrer Vergehen als 
Piroggenanführerin auf Sankt Krokant verhaftet. Eclair, Otto und die Mannschaft der Speckkugel versuchen sie mit allen Mitteln zu 
befreien, letztendlich bringt der aus dem Kloster zurückgekehrte Mordan die glückliche Wende.

\subsubsection{Ausgangs- und Endsituation}
Die Ausgangssituation wird im Prolog und in den Kapiteln eins bis drei beschrieben. 
Zunächst wird von den Piroggen, ihrer Herkunft und ihren Eigenschaften erzählt und von der 
Handelsstadt Murseille berichtet. Murseille ist der Ausgangsort von Mohnschnecke und Eclair. 
Sie wohnt dort bei ihren wohlhabenden Eltern, die ein Mohngeschäft betreiben und er arbeitet dort 
für ihren Vater. Das Mädchen möchte einen Ball besuchen von dem sie schon lange träumt und fragt ob 
er sie nach Sankt Krokant rudern könne. Dort treffen sie auf Hörnchen, der das Mädchen einlädt einen 
Ausflug auf seiner Yacht zu machen. Alles ändert sich, als sie von wilden Piroggenpiraten entführt wird 
und Eclair und Hörnchen sich auf die Suche nach ihr machen.
\\
Am Ende muss Mohnschnecke, (Kapitel 59 bis Epilog) die nun Käptn Mohnschnecke ist, nicht mehr von den Piroggen, 
sondern aus dem Gefängnis der Stadt Murseille befreit werden. Dorthin war sie gebracht worden nachdem sie von 
Halbpirogg und Hörnchen verraten worden war und wartet nun auf die Todesstrafe durch Verbrennen. Eclair 
versucht sie zusammen mit der Mannschaft der Speckkugel zu befreien, doch erst als Mordan plötzlich auftaucht, 
gelingt es allen beteiligten Piraten zu fliehen. Zurück auf dem Schiff, wird Halbpirogg verbannt, Mordan wieder 
Käptn und Mohnschnecke und Eclair verlassen gemeinsam die Piroggen. Auf Graf Napoleons Frühlingsball macht Eclair 
dem Mädchen einen Heiratsantrag und die beiden sind nun mit „einer kleinen aber treuen Schwarzpiroggenmannschaft“ 
unterwegs. (S.643) Mohnschneckes Vater wird noch reicher, als er unerwartet, möglicherweise von seiner Tochter, 
mit einer großen Kamelherde beschenkt wird, die „Mohnsaat aus Damaskus“ \cite[S.645]{pir} geladen hat.
Hörnchen heiratet die Zimtschnecke, die er auf dem Markt von Djadida für Mohnschnecke gehalten hatte \cite[S.644]{pir} 
und Käptn Mordan entdeckt mit seiner Mannschaft einen neuen Kontinent. (S.646)


\subsubsection{Handlungsbestimmende Ereignisse}

Jeder der drei Hauptprotagonisten, sowie Halbpirogg und Käptn Mordan erleben Ereignisse die die Geschichte 
grundlegend beeinflussen. Das Schicksal von Mohnschnecke hat Auswirkungen auf den gesamten Handlungsverlauf: 
Ohne ihre Entführung könnte sich die Geschichte nicht so Anders von der Ausgangssituation entwickeln: keiner 
der Beteiligten hätte einen Grund nach ihr zu suchen. Die Neuerungen an Bord und ihre persönliche Entwicklung 
beeinflussen Halbpirogg negativ und Mordan positiv. Mordans Heiratsantrag in Kapitel 31 führt dazu, dass Käptn 
Mordan ins Kloster geht, sie Kapitän auf der Speckkugel wird, damit aber auch ihre gute Stimmung auf dem Schiff 
kippt und sie beginnt sich und ihr Handeln zu hinterfragen, was auf längere Sicht zu ihrem emotionalen Tiefpunkt 
im Gefängis führt. (Kapitel 57) Halbpiroggs andauernde Unzufriedenheit gegenüber Mohnschnecke führt schließlich zu 
seinem Racheplan, sodass er zusammen mit Hörnchen, das Mädchen verrät und festnehmen lässt. (Kapitel 55) 
Die zunächst kleine Auseinandersetzung mit den Spuniern in Kapitel 26, schaukelt sich über Kapitel 28, 
der „Flucht à la Venezia“, bis zur „Schlacht von Trüffelgar“(Kapitel 43) hoch. Die Mannschaft der Speckkugel 
und auch Kapitän Li, der  das Reispiroggeschiff auf dem sich Eclair befindet befehligt, erfahren von den 
Pilzpiroggen, dass sich alle Piroggen zu einer Schlacht zusammenfinden (Kapitel 32 und 39)
Eclair trifft nach der Verbannung von seiner Wohn- und Arbeitsstelle in Murseille auf ein Pelmenidorf (Kapitel 5), 
dort wiederum lernt er Otto kennen, mit dem zusammen er eine Brigantine repariert, die ihm die Möglichkeit eröffnet 
auf See nach Mohnschnecke zu suchen. Durch den Sturm in Kapitel 22 wird zwar ihr Boot zersört, dadurch werden sie 
jedoch von den Reispiroggen aufgenommen und segeln mit zur Schlacht bei Trüffelgar. Obwohl Eclair jetzt direkt vor 
der Gesuchten steht, schickt sie ihn und seinen Freund nach Hause. Hätte sein Begleiter und Freund Otto, der Pelmen, 
nicht schon vor der Reise den dringenden Wunsch ein echter Pirogg zu werden, würde ihn Eclair nicht zum 
Piroggenbackofen begleiten, (Kapitel 49) weswegen Otto auf der Speckkugel arbeiten kann (Kapitel 52) und somit 
würde Eclair wohl nicht erfahren, dass Mohnschnecke in Schwierigkeiten ist (Kapitel 56). Ohne diesen Umstand könnte 
er sich ebensowenig mit der Mannschaft der Speckkugel verbünden, um Mohnschnecke zu retten. (Kapitel 56) Hörnchen 
wird erst von seinem Vater dazu gedrängt nach Mohnschnecke zu suchen (Kapitel 10). Die Verkleidung, die er und 
Zwieback bei ihrer Flucht von der Burg des Vicomte de Brinson, (Kapitel 25) von Alalie erhalten, hilft ihnen zwar 
über die Grenze nach Wursterreich, lässt sie aber als „Spione“ auffliegen, sodass sie ins Gefängnis kommen (Kapitel 30) 
Die Bekanntschaft mit der Knofikarawane (Kapitel 14) führt zwar dazu, dass der Knofiälteste ihm zu einem Freispruch 
verhilft, der Vicomte de Brinson legt jedoch Einspruch ein: es kommt zu einem Turnier. (Kapitel 33) 
Nachdem Hörnchen das Turnier gewonnen hat (Kapitel 35), fährt er mit Zwieback nach Djadida, wo sie eine vermummte, 
vermeintliche Mohnschnecke kaufen (Kapitel 38), die sich zu Hause als Zimtschnecke herausstellt. (Kapitel 50) 
Die dadurch erfahrene Ablehnung in seiner Heimatstadt (Kapitel 50) macht ihn empfänglich für Halbpiroggs 
Racheplan. (Kapitel 54) Mordan erfährt im Kloster von Mohnschneckes Verhaftung und dem nahenden Ende der 
Piroggenherrschaft (Kapitel 61), sodass er schließlich als Mönch die ausschlaggebende Rolle bei ihrer 
Befreiung spielt. (Kapitel 64)


\section{Charaktere}
Im Laufe der Handlung lernt man viele unterschiedliche Charaktere kennen. Im Folgenden werden Hauptfiguren,  Nebenfiguren und wichtige Personen angesprochen.

\subsection{Hauptfiguren}

Mohnschnecke ist die Hauptprotaginistin des Buches 'Die wilden Piroggenpiraten' und die gesamte Geschichte dreht sich um sie. Daneben werden die Ereignisse um Eclair und Hörnchen beschrieben, sodass sie ebenso zu den Hauptfiguren zählen.

\subsubsection{Mohnschnecke}

Die Hauptfigur ist „eine an Jahren gänzlich junge und schöne Mohnschnecke“ \cite[S.13]{pir}
Ein „sehr sentimental(es)“ Mädchen (S.15), das  zwar aus gutem Hause kommt (S.14) „viele Romane (liest)“ 
und dementsprechend gebildet (S.15), sich ihrer gesellschaftlichen Stellung aber durchaus bewusst ist: 
Als Eclair sie mit „Guten Morgen Mademoiselle!“ begrüßt und „sich höflich (verbeugt)“, wird sein Gruß nicht 
erwidert. (S.16). Unhöflich blafft sie ihn bei der Frage wieviel er abwiegen dürfe an
und meint 'Bah, abwiegen – also nein, was denken Sie sich denn? (…)(und nimmt ihm) kokett das Schäufelchen aus der Hand'. \cite[S.16]{pir}
Zu Beginn ist sie manipulativ und blauäugig denn, '(d)ass sie Eclair gefiel war für Mohnschnecke kein Geheimnis.'  \cite[S.17]{pir}
' (…) dem Mädchen (kam) einer von diesen unschuldigen Einfällen in den Sinn, die manchmal völlig unvorhersehbare und weitreichende Folgen zu haben geruhen' \cite[S. 19]{pir}
und bringt ihn dazu sie trotz seiner Zweifel nach Sankt Krokant zu fahren.  \cite[S. 19]{pir}

Sie ist egoistisch, ignorant und hochnäsig \cite[S.23]{pir}, als sie mit Eclair zusammen auf Sankt Krokant ankommt, bemerkt sie nicht, dass er, der der die ganze Zeit gerudert ist einen schmerzenden Rücken, wunde Hände \cite[S.23]{pir} und 'die (zitternden) (Finger) eines alten Zwiebacks (hat)'\cite[S.23]{pir}, stattdessen meint sie 'Sie Rüpel hätten mir ruhig mal die Hand reichen können'\cite[S.23]{pir} und springt dann ohne Probleme von alleine 'auf die Planken der Anlegestelle'.\cite[S.23] 
Vor ihrer Gefangennahme „sehnte sich (Mohnschnecke), 
je älter sie wurde, zunehmend nach aufregenden Abenteuern“ \cite[S.15]{pir}, sodass sie trotz ihres heftigen Widerstands 
bei der Gefangennahme durch die Piroggenpiraten \cite[S.32]{pir} mit der Zeit immer mehr Gefallen am Leben auf der Speckkugel 
findet. Sie ist mutig beim Angriff auf ein venezianisches Handelsschiff, \cite[S.52]{pir} begeistert von der Ausbeute, die sie 
bei den Raubzügen machen \cite[S.59ff]{pir}, trägt nun Hemd und Hose\cite[S.80]{pir} und zeigt sich ehrgeizig beim Fechtunterricht. \cite[S.127]{pir} Zwar 'hatte sie es (a)nfangs schwer (...) und dem Mädchen (taten) abends die Arme unerträglich weh. Aber mit der Zeit verwandelten sich die Schwielen an den Handflächen in harte Hornhaut, (und) die Arme wurden immer geübter (...).'\cite[S.127]{pir. Zwar führt Käptn Mordan wegen ihr einige Neuerungen in Sachen Hygiene \cite[S.144ff]{pir}und 
Morgengymnastik \cite[S.180f]{pir} ein, die der Mannschaft  zusammen mit der Veränderung Morgans sehr eigenartig 
vorkommen,\cite[S.94]{pir}\cite[S.129]{pir} mit Hilfe ihrer Ideen, ihrem Mut und ihrer schnellen Auffassungsgabe  hilft sie 
den Piroggen aber auch mehrere Male aus brenzligen Situationen z.B. als Mordan angegriffen wird\cite[S.100]{pir}  und 
als sie die Spunier \cite[S.254ff]{pir}austrickst. Dafür erntet sie die Anerkennung der Mannschaft nachdem sie ihr aus der scheinbar auswegslosen Situation, beim Angriff durch eine spunische Fregatte eingeschlossen in einer Lagune hilft,\cite[S.260]{pir} 'Du bist eine echte Pirogge!' sprach Mario stellvertretend den Gedanken aller aus (...)'.\cite[S.260]{pir}
und befindet sich vorerst auf einem emotionalen Höhepunkt.  Nach  Käptn Morgans Heiratsantrag,\cite[S.297ff]{pir} den sie 
ablehnt,\cite[S.302]{pir} woraufhin er beschließt in ein Kloster zu gehen, beginnt sie jedoch an ihrem Leben auf dem 
Piroggenschiff zu zweifeln,\cite[S.302]{pir} '(d)as Leben kan ihr mit einem Mal fürchterlich kompliziert vor - alles war so schön gewesen, aber jetzt...'.\cite[S.302]{pir} Jetzt betrachtet sie ihr früheres Ich kritisch, erinnert sich an '(i)hr Zuhause, den friedlichen Gewürzladen, die Romane, die sie früher gelesene hatte (und an) den hilfsbereiten Verkäufer Eclair - das alles (kommt) ihr so weit weg vor, als hätte irgendein anderes, naives und romantisches Mädchen in einem vollkommen anderen Leben das alles erlebt...'.'\cite[S.379]{pir} Ihre 
Unsicherheit verbirgt sie und  wandelt sie stets in einen anderen Gemütszustand um: als Käptn Morgan das Schiff 
verlässt wird sie wütend (S. 375), als Bonaventura ihr vor der Abstimmung zum neuen Kapitän zur Flucht verhelfen 
will, weil es gefährlich für sie werden könnte, reagiert sie stolz und hartnäckig.\cite[S.378f]{pir} Auch Eclair fällt 
die Veränderung auf, als sie sich nach der Schlacht von Trüffelgar auf der Speckkugel treffen (S.461). Zwar meint 
Mohnschnecke dabei traurig„(d)iese Mohnschnecke gibt es nicht mehr“\cite[S.463]{pir} und bekennt sich dadurch zu ihrem Wandel, 
will diesen aber noch nicht wahrhaben und schickt Eclair und Otto weg.\cite[S.463f]{pir} Als Halbpirogg sich als schlechter 
Koch erweist,\cite[S.218ff]{pir} gesteht sie sich zwar ein, dass sie ihr altes Leben vermisst, schüttelt den Gedanken 
zunächst ab,\cite[S.521f]{pir} fragt die Mannschaft, wohl auch weil sie sich ihrer Verantwortung als Kapitän bewusst ist, 
aber letztendlich ob sie nach Sankt Krokant fahren könnten,\cite[S.529]{pir} Auf Napoleons Ball wird sie festgenommen \cite[S.551]{pir}. 
Auch wenn sie bei der Verhaftung gehäßig reagiert und Hörnchen ironisch fragt ob ihm der Spaziergang über die Planke gefallen habe \cite[S.522]{pir} und bei der Gerichtsverhandlung „stolz und 
unbeugsam“ \cite[S.573]{pir} auftritt, ist sie dennoch hoffnungslos und befindet sich im Gefängnis auf ihrem emotionalenTiefpunkt.\cite[S.561]{pir}Als ihr Vater ihr anbietet an ihrer Stelle im Gefängnis zu bleiben lehnt sie ab \cite[S.562]{pir}. Bei der Befreiung fällt sie in Ohnmacht – „zu viel hatte sie in letzter Zeit durchmachen müssen“\cite[S.617]{pir}. Als sie wieder aufwacht ist sie liebevoll zu Eclair und „(streicht) leicht über (seinen) Kopf“, später erklärt sie ihm den Nachthimmel.\cite[S.631]{pir}Mordan möchte bevor sie geht ihren Lohn auszahlen, dabei reagiert sie Bescheiden und lehnt das erst einmal ab, wobei man bei ihrem Gedankengang ihr Verantwortungsbewusstsein  erkennt: \cite[S.638]{pir} '(s)ie hatte einfach getan was getan werden musste'.\cite[S.638]{pir} Zum Abschied 'küsst (sie) den alten Mann auf die Wange.“\cite[S.638]{pir}

\subsubsection{Eclair}

Eclair ist ein höflicher,\cite[S.16]{pir}\cite[S.42ff]{pir}„sehr gut aussehender Jüngling“.\cite[S.15]{pir} Er ist gebildet, \cite[S.15]{pir} musste aber trotz seiner '(l)ängere(n) Studien an der Bonbonne (wo er) seine geistigen Fähigkeiten vervollkommnet (hat)'\cite[S.15]{pir} aufgrund seiner finanziellen Lage sein Heimatland verlassen.\cite[S.15f]{pir} Er arbeitet für Mohnschneckes Vater und verheimlicht ihr bei der Frage ob er rudern könne \cite[S.17]{pir} seinen Misserfolg, da sie ihm gefällt.\cite[S.17]{pir} Als er beobachtet, wie Hörnchens Yacht von Piroggen geentert wird, denkt er nur an Mohnschnecke und will sie unbedingt retten\cite[S.29]{pir} auch wenn „(er) keine Ahnung (hatte), was er tun sollte, wenn er das Piroggenschiff erreichen würde und wie er Mohnschnecke helfen könnte.“ Da er es nicht schafft sie zu befreien und weil er Hörnchen für die Entführung und den möglichen Tod Mohnschneckes verantwortlich macht, reagiert er, wohl aus Hilflosigkeit, aber auch aus Ärger über Hörnchen, zornig.\cite[S.36]{pir} Er wird in das Dorf aufgenommen,\cite[S.67]{pir}, auf das er nach seiner Nacht am Strand trifft, \cite[S.43]{pir} zeigt sich hilfsbereit,\cite[S.44]{pir} bietet an zu arbeiten\cite[S.44]{pir} und sozial indem er Otto 'zu einem Satz erstklassiger Segel '\cite[S.106ff]{pir} verhilft. Zudem ist er bescheiden und er 'baut sich im Laufe der Zeit eine kleine Hütte aus Treibgut ((a)m Rande der Pelmenisiedlung),\cite[S.67]{pir} so '(haben) sich (die übrigen Einwohner) schnell an den ungewöhnlichen Nachbarn gewöhnt'.\cite[S.67]{pir} Als er mit Otto in See sticht, bemerkt er selbst, dass er sich verändert hat, sowohl äußerlich, die 'Haut (der) Handflächen\cite[S.66]{pir} des vorher untrainierten wenig muskulösen Jungen \cite[S.15]{pir} '(werden) vom täglichen Rudern(...) allmählich dick und hart,\cite[S.66]{pir} wie auch persönlich, '(der Eclair von früher) (scheint) ein völlig anderer zu sein (...), der vermeintlich rein gar nichts gemein hatte mit dem heutigen (...) - dem Elritzenfischer aus dem Pelmenidorf und Schiffsbaumeister, mit den schwieligen Händen und kräftigen Muskeln'\cite[S.137]{pir} Beim lenken der Brigantine, zusammen mit Otto zeigt er sich teamfähig. \cite[S.134]{pir} Er ist schlau, was sich bei seinem Angriff auf ein Grützschiff zusammen mit den Reispiroggen zeigt. \cite[S.309ff]{pir} Beim Gedanken an eine Begegnung mit der Speckkugel ist er  unsicher und ängstlich.\cite[S.162]{pir} Nach einem Sturm, als Eclair mit Otto im Meer treibt, schließt er aus Verzweiflung und Perspektivlosigkeit über seine Lage schon mit seinem Leben ab. \cite[S.209]{pir} Erst als Nachdem er noch ungläubig endlich auf der Speckkugel gelandet ist,\cite[S.460]{pir} wird er von Mohnschnecke nach Hause geschickt \cite[S.464]{pir} und erlebt  nach einem ersten Tiefpunkt, bei dem er angesichts seines Vorhabens unsicher ist \cite[S.414]{pir} nach dem Abschied von Otto die Einsamkeit und Traurigkeit \cite[S.515f]{pir} Zurück im Pelmenidorf 'war ((d)er Himmel) immer häufiger bewölkt, und manchmal wüteten auf dem Meer richtige Stürme. Dann blieb Eclair an der Küste und starrte stundenlang auf die Wellen.'\cite[S.515f]{pir} Mit Mohnschnecke hat er abgeschlossen und aus dieser Lage löst er sich erst, als ihn Otto um Hilfe bei der Befreiung von Mohnschnecke bittet und ihm sagt „Ja, es ist wichtig für mich“ (S.556) Auch wenn ein erster Befreiungsversuch scheitert, gibt Eclair die Hoffnung nicht auf und es offfenbart sich, dass er in Mohnschnecke verliebt ist. \cite[S.603]{pir}

\subsubsection{Hörnchen}

Hörnchen „(sieht) umwerfend aus“ (S.25) und „hat(te) eine geschmeidige, kräftige Figur“ (S.25). Er ist charmant und 'zwinkert (Mohnschnecke) zu und (legt) zum Gruß zwei Finger an die Mütze (a)ls er Mohnschneckes Interesse bemerkte)'\cite[S.25]{pir}, aber hochnäsig und egoistisch: er erzählt sofort von seinem bekannten Vater\cite[S.25]{pir}, denkt Eclair sei Mohnschneckes Diener\cite[S.26]{pir} und denkt mit Geld kann er sich von den Piroggen freikaufen, als '(er kräht:) Lasst mich laufen! Mein Vater ist reich, er wird bezahlen!...'.\cite[S.34]{pir} Obwohl er sich auf der Speckkugel nur um seine eigene Befreiung bemüht hat, behauptet er heldenhaft gekämpft zu haben.\cite[S.40]{pir} Er ist faul und drückt sich vor Verantwortung, täuscht vor immernoch verletzt zu sein (S.62 f) und begibt sich erst auf die Suche nach Mohnschnecke, als ihm sein Vater Anweisungen erteilt (S.82 ff). Beim Überschreiten des Grenzübergangs nach Käsien geht sein persönlicher Stolz, über das Allgemeinwohl und er reagiert missmutig auf den Knofiältesten.\cite[S.142f]{pir}
In einer Pizzeria ist er sich zu fein im Rahmen des Reisebudgets zu essen, widersetzt sich Zwieback und denkt respektlos über ihn\cite[S.170f]{pir}, denn '(er) (macht) was er will! Der alte Knochen muss so oder so für (ihn bezahlen)',\cite[S.170f]{pir} widersetzt sich Zwieback und denk respektlos über ihn \cite[S.170f]{pir}, er lässt sich ungern Ratschläge geben\cite[S.170f]{pir}\cite[S.219f]{pir}Als ihm die Pizza Knallpuffer anbietet und er wie selbverständlich bestellt, ist er zu stolz zuzugeben, dass er diese nicht kennt und lässt sich von den Gästen und der Bedienung bei seiner Bestellung schmeicheln und leicht beeinflussen. (S.171) Später gibt er trotz seines eigenen Fehlverhaltens Zwieback die Schuld. (S219) Dass er trotz seiner stolzen Art wenig Selbstbewusstsein hat zeigt sich, als die Käsewurst nach der Schlacht mit den Blutwürsten missverständlicherweise annimmt, dass Hörnchen sie zur Frau nehmen will (S.230), statt das Missverständnis aufzuklären flüchtet er mit Zwieback durch ein Fenster (S.233 ff). Aber hier zeigt sich zum ersten Mal Mitgefühl, er ist Verzweifelt und „kann (…) dem armen Mädchen doch nicht mehr sagen, dass (er) sie nicht mehr heiraten will...“ (S.239) Als Hörnchen und Zwieback in Wursterreich als vermeintliche Spione enttarnt werden und auf dem Weg ins Gefängnis „mit ranzigen Speckstücken (beworfen)“ (S.294) ist er „am Boden zerstört.“ (S.294) Im Gefängnis verliert er zunächst seinen Stolz und isst Gerstengrützbrei  dann lässt sein Widerstand nach, als er Zwieback schimmeln sieht. (S.322) Dass Hörnchen eine eher langsame Auffassgabe hat, zeigt sich nach dem Vorkommnis an der Grenze zu Käsien (S.142), nun auch nachdem er und Zwieback freigesprochen werden: Zwieback versucht aus der Situation Gewinn zu schlagen und Hörnchen fragt „Hatten wir wirklich vierhundert Taler?“ (S.331) Seinen Stolz hat er auch in der Gefängniszelle nicht verloren, wie sich bei der Gerichtsverhandlung, \cite[S.336]{pir}, als 'der Richter (beginnt)'Dann wird der Weißbrotbursche...' und Hörnchen (ihn unterbricht:)'Hörnchen, mit Verlaub. Mein Name ist Hörnchen'.\cite[S.336f]{pir} und später auf dem Schiff der Paprikapiroggen zeigt.\cite[S.367f]{pir} Aber sein Mitgefühl ist gewachsen, er verspricht Alalie, ihren Vater nicht zu verletzen (S.341) und freut sich beim Turnier über die Anwesenheit von Leuten die er kennt.(S.349) Dass eigentlich die falsche Mohnschnecke in Djadida gekauft wurde, ist Hörnchen eigentlich bewusst, aber er scheint das schnell zu vergessen und nach Hause fahren zu wollen. (S.390) Als sich dann herausstellt, dass es sich um eine Zimtschnecke handelt und Hörnchen zum Gespött der Stadt wird, bemitleidet er sich ebenso wie am Anfang selber. (S.514) zieht sich zurück und bereut Alalie nicht geheiratet zu haben. (S.538) Als ihm Halbpirogg seinen Racheplan erzählt, sieht er eine Möglichkeit wieder an im Ansehen der Stadtbewohner zu steigen. (S.540) Allerdings: „Warum fühlte er sich dann so schlecht?“ (S.559)

\subsection{Wichtige Nebenfiguren}
Neben den Hauptfiguren spielen Mordan, Halbpirogg und der Pelmen Otto eine wichtige Rolle in der Geschichte.

\subsubsection{Mordan}
Cristóbal Mordan ist der Kapitän der Speckkugel,\cite[S.636]{pir}'(hat) ein Holzbein (und trägt) eine schwarze Augenklappe, einen mit befransten goldenen Epauletten verzierten schwarzen Uniformrock'\cite[S.31]{pir} und einen  Krummsäbel und zwei silberne Pistolen bei sich. \cite[S.31]{pir} Er kümmert sich von Anfang an um Mohnschnecke und zweifelt nie an ihrer Anwesenheit auf dem Schiff, liebevoll nennt er sie schon nach kurzer Zeit 'mein kleines Feingebäck'.\cite[S.61]{pir} Zwar werden Gefangene üblicherweise auf dem Markt in Djadida verkauft \cite[S.50]{pir} aber Mordan denkt nicht einmal daran sie zu verkaufen, im Gegenteil:  er lässt sie zu ihrem Schutz vom Mast losbinden, als die Speckkugel in ein Gefecht gerät, \cite[S.52f]{pir} überlässt ihr die Kapitänskajüte und schläft in einer Hängematte\cite[S.77ff]{pir} und fängt an sich zu waschen.\cite[S.129]{pir} Er zeigt ihr gegenüber immer mehr Vertrauen, etwa nach dem Angriff auf eine portugiesische Galleone \cite[S.94ff]{pir}, als er ihr '(v)erlegen (...) (ein) Kästchen (mit zwei wundervolle(n), mit Perlmuttgriffen und prachtvollen Goldintarsien verzierte(n) Duellpistolen (überreicht) \cite[S.102]{pir} und es zeigen sich erste Anzeichen von Sympathie und Verliebtheit: 'Nur wenn Mohnschnecke auf Deck erschien (…), unterbrach Mordan sein Gefluche'\cite,[S.94]{pir} er wird nervös wenn er sich verspricht,\cite[S.102]{pir} macht ihr Geschenke,\cite[S.102]{pir} bewahrt drei Mohnkörnchen von ihr auf \cite[S.182]{pir} und weicht Fragen zu ihrem Verkauf entweder aus oder reagiert gereizt.\cite[S.61296]{pir}\cite[S.145]{pir} Die Neuerungen, die er wegen ihr einführt sieht er positiv, 'denn eine saubere Mannschaft sah viel fescher aus und litt auch seltener an Schimmelschnupfen oder verdorbener Füllung'\cite[S.181]{pir} Als Halbpirogg versucht die Mannschaft gegen sie aufzubringen, geht er nicht weiter auf die Anschuldigungen ein und macht seinen Standpunkt klar, indem er Halbpirogg schlägt.\cite[S.188]{pir} Er verbringt viel Zeit mit Mohnschnecke, erklärt ihr die Sterne (S.207) zeigt ihr den Umgang mit dem Sextanten.\cite[S.297]{pir} Als er ihr einen Heiratsantrag macht,ist er nervös und versucht ihr deswegen zunächst Komplimente zu machen, doch auch als er sich überwinden kann,  lehnt sie ab \cite[S.300]{pir} und er ist am Boden zerstört, sodass er seine Kajüte vier Tage lang nicht verlässt.\cite[S.302]{pir} Verletzt und weil er einsieht, dass Mohnschnecke der tapferere, bessere Kapitän \cite[S.300]{pir} und jünger ist als er \cite[S.371]{pir} verlässt er das Schiff und überträgt ihr die Verantwortung. Er flieht in die Einsamkeit eines Klosters \cite[S.400ff]{pir} 'um Mönch zu werden, Frieden und Erleuchtung zu finden und sein ganzes bisheriges Leben zu vergessen - und gewisse Mädchen mit Mohn auf der Nase...'\cite[S.469]{pir} und kann Mohnschnecke dennoch nicht vergessen, die Dose mit 'drei winzigen Mohnkörnchen'\cite[S.632f]{pir} hat er immernoch. Es bereitet ihm Schwierigkeiten sich ans Klosterleben zu gewöhnen, zeigt sich uneinsichtig und kann seine Piroggeneigenschaften nicht ablegen. (S.467) Sein Widerstand schwindet jedoch als er eingemauert wird hoffnungslos, traurig und reumütig,\cite[S.582ff]{pir}'der Anblick der winzigen schwarzen Körnchen erschütterte ihn so sehr, dass Mordan das kleine Behältnis wieder in der Tasche verstaute und nicht wieder hervorholte.'\cite[S.531]{pir} Als er jedoch von ihrer Hinrichtung erfährt , überwindet er alle Widerstände und rettet sie \cite[S.620]{pir} Zwar hegt er immernoch eine gewisse Zuneigung zu ihr, akzeptiert aber, dass sie mit Eclair zusammen ist. \cite[S.632f]{pir}

\subsubsection{Halbpirogg}
Halbpirogg ist der Kapitänsmaat. Seinen Namen hat er seit ihm ein Gegner bei einer Schlacht 
einen beträchtlichen Happen aus seiner Seite gesäbelt hatte (...)(dort) 
war ihm ein Teil der Füllung herausgepurzelt (...) 
(und) (dort wurde er) mit Hanfgarn zugenäht\cite[S. 33]{pir}
Er bemerkt als Erster die Veränderungen des Kapitäns nach 
dem Auftauchen Mohnschneckes, stichelt und ist ihr gegenüber pessimistisch:
er redet ihre Vorschläge schlecht \cite[S. 266]{pir} schimpft immer wieder über Diese verflixte Mädel\cite[S. 145]{pir}
die verflixten Weiber \cite[302]{pir}und 
hegt ihr gegenüber eine negative Grundeinstellung. \cite[S. 297]{pir}
Als er Käptn Morgan daran erinnert, dass Mohnschnecke, wie es eigentlich üblich ist \cite[S.50]{pir} 
auf dem Markt verkauft werden soll (S.129), stößt dabei immer wieder auf wenig Aufmerksamkeit. \cite[S. 145]{pir}
Seine schlechte Laune ihr gegenüber, ist in Käptn Morgans Aufmerksamkeit gegenüber Mohnschnekce begründet: er macht
sich Hoffnungen irgendwann Käptn Morgans Nachfolge anzutreten, 
da er (a)ls ältestes Besatzungsmitglied (...) die meiste Erfahrung (...) (hatte)\cite[S.371]{pir}
 
Da er (e)ine Menge Mannschaftsmitglieder  (...) im Lauf der Zeit gegen sich aufgebracht (hatte), \cite[S.151]{pir}, 
stürzen die sich beim Waschen mit Vergnügen auf ihn \cite[S.151]{pir}
Aus Frust über die mangelnde Anerkennung und die Veränderungen an Bord, 
(stattet) er in letzter Zeit immer häufiger den im Laderaum gestapelten Rumfässern 
einen Besuch (ab) \cite[S. 181]{pir} und sucht diese immer wieder auf, wenn er sich ärgert \cite[S. 303]{pir}


\subsubsection{Pelmen Otto}


(\subsection {Personenkonstellationen}
Die Beziehungen der Charaktere untereinander spielen im Laufe der Handlung eine wichtige Rolle. So verändern sich nicht nur die Relationen zwischen Mohnschnecke und Mordan, Mohnschnecke und Eclair, Eclair und Otto und Hörnchen und Zwieback, sondern dadurch auch die Handlung.
\subsubsection {Mohnschnecke und Mordan}
\subsection{Mohnschnecke und Eclair}
\subsubsection {Eclair und Otto}
\subsubsection{Hörnchen und Zwieback}
Der anfängliche Unmut der Mannschaft gegenüber Mohnschnecke und den Veränderungen, die mit ihr auf der 
Speckkugel einkehren \cite[S. 77 f]{pir} verfliegt, als die Piroggen Gefallen an )

\subsection{Schauplätze}
Die Schauplätze der Handlung lassen sich in die Ursprungsorte der Protagonisten, ihre Verweilorte und wichtige Schauplätze unterteilenn.

\subsubsection{Ursprungsorte}

Murseille ist ein ‚kleine(s) (eintönig)(s) Hafestädtchen, in dem außer den regelmäßigen 'Wahlen und Messen (…) Jahr für Jahr‘ \cite[S. 10[{pir} nichts außergewöhnliches passiert. Die Bewohner leben sicher ‚in häuslicher Geborgenheit‘ und erzählen sich höchstens Gruselgeschichten von ‚wilden Piroggenpiraten, (die als) etwas Fernes und Sagenhaftes (gelten)‘ \cite[S.10]{pir} und in einer nicht greifbaren Welt leben. Gerüchten zufolge hat der Onkel Mohnschneckes Vaters schon ‚mit Großvater Klatschmohn heimlich große Mengen (in Murseille strengstens verbotener) Mohnmilch (importiert)‘\cite[S.14]{pir} Mohnschneckes Mutter, eine Mohnpotitze, kommt aus Wien in Austerreich. \cite[S. 13]{pir} Sankt Krokant, eine Insel, die sich im Meer vor Murseille befindet, ist der Austragungsort des Frühlingsfestes zu dem Mohnschnecke möchte \cite[S.17]{pir}, auf dem ‚sie genaugenommen nur ein einziges Mal als kleines Kind gewesen war und ihren Eltern zugesehen hatte, (…) (auf Graf Napoleons Fest) tanzten‘ \cite[S.521]{pir} 
Hörnchen ist ebenso hier ansässig, auch wenn er sich mit seiner Yacht oft auf Sankt Krokant aufhält, denn auf Mohnschneckes Zögern auf sein Angebot, eine Runde mit seiner Yacht zu drehen antwortet er unter Anderem: ‚(W)ir (benehmen) uns auf Sankt Krokant alle ein bisschen ungezwungener.‘\cite[S.25]{pir} 
Über Eclair erfährt man nur, dass er hat aufgrund seiner finanziellen Lage sein Heimatland verlassen und zuvor an der Bonbonne studiert hat, somit könnte er aus dieser Region kommen. \cite[S.15f]{pir}
Die Piroggen haben unterschiedlichste Herkunftsorte: die Wan-Tan-Piroggen stammen aus Tai-Wan-Tan, \cite[S.8]{pir}
 ‚die bärtigen Piroggen aus Sibirien' \cite[S.8]{pir}, ‚die Belaschis aus dem hohen Norden \cite[S.9]{pir} und die gehörnten Pilzpiroggen aus Stulle (im) hohen Norden‘\cite[S.9]{pir}.Über die Kaldaunenpiroggen, die Fischpiroggen, die Krümelquarkpiroggen und die Speckpiroggen, zu denen Morden und seine Mannschaft zählen, erfährt man zunächst keine genauere Herkunft. \cite[S.9f]{pir}

Bei einer genaueren Beschreibung der Speckkugelmannschaft erfährt man, dass Bubba ‚von den Safraninseln im fernen Süden‘ \cite[S.46]{pir}, Indrick aus Lettland \cite[S.46f]{pir}, die 'drei pechschwarze(n) Piroggen (…) Tschonbé, Gambia und Fu (…) (vom) heißen südlichen Erdteil‘ \cite[S.47]{pir}, ‚Pedro aus Spunien'\cite[S.47]{pir}, 'Sa-Awedra aus dem Biskenland' \cite[S.47]{pir} und Ribeira aus Portugal \cite[S.47]{pir} stammen.


\subsubsection{Verweilorte}
Verweilorte bezeichnen Umgebungen in denen sich die Protagonisten längere Zeit aufhalten.

Mohnschnecke verbringt ab ihrer Entführung durch die Speckpiroggen \cite[S.32]{pir} die meiste Zeit an Bord der Speckkugel, wo sie in der Kapitänskajüte schlafen kann, nachdem sie mit Mordan darüber diskutiert hatte, 'ob sie bei den Matrosen schlafen (soll)' \cite[S.78]{pir}, sie 'mit (k)einem unbekannten Mann im selben Raum schlafen (kann),\cite[S.78]{pir}und dass es keine Damentoilette gibt, \cite[S.78]{pir}, sodass es Mordan schließlich irgendwann reichte, \cite[S.78]{pir} 'er (...) fluchtartig seine Kapitänskajüte (verließ)'.\cite[S.78]{pir} Dort nimmt sie Fechtunterricht \cite[S.127f]{pir} und führt Körperhygiene und Morgengymnastik ein - 'eine Neuerung auf die sie ziemlich stolz ist \cite[S.181]{pir} und über die 'auch Käptn Mordan zufrieden (ist), denn eine saubere Mannschaft sah viel fescher aus und litt auch seltener an Schimmelschnupfen oder verdorbener Füllung.'\cite[S.181]{pir} Ausflüge vom Schiff unternimmt sie selten,zuerst, als Mordan die Stadt Madeira angreift \cite[S.186ff]{pir} damit Mohnschnecke zum Friseur gehen kann \cite[S.195]{pir} und als die Speckkugel 'in (einer) Lagune vor Anker (geht)'.\cite[S.247]{pir} Diese gehört zu einer Insel, 'die aus einem vor Urzeiten erloschenen Vulkan entstanden war',\cite[S.246]{pir} dort sollten Reperaturarbeiten nach einem Gefecht durchgeführt werden.\cite[S.245f]{pir} Da ihr 'die Geschäftigkeit der Zimmerleute und das laute Gehämmer bald auf den Keks (gehen)'\cite[S.247]{pir} fragt sie um ein Boot um zur Kraterwand zu fahren, die sie hinaufklettert und von oben eine gute Aussicht hat. \cite[S.247f]{pir} Beim Aufstieg erfährt sie 'beim Aufstieg (...) das Gefühl, als würde die Erde leicht schwanken'\cite[S.248]{pir} und bemerkt selbst, 'dass dieser Eindruck lediglich eine Folge der langen an Bord verbrachten Zeit war. Ihr Körper war so sehr an das ununterbrochene Schaukeln des Schiffes gewöhnt, dass das vollkommen unbewegte Festland irritierend wirkte.'\cite[S.248]{pir} Als dort 'drei Fregatten der spunischen Armada (...) auf dem Weg sind'\cite[S.250]{pir} hat sie eine Idee \cite[S.251]{pir}: Sie rudert in einem Boot zu den angreifenden Schiffen, bittet um Hilfe und lässt sich an Bord holen. Dort erzählt sie dem Kapitän '(d)ie bösen Piroggen (hätten) sie entführt, doch es (wäre ihr gelungen) zu entkommen, als sie sich betrunken auf dem Deck herumwälzten'.\cite[S.254]{pir} und dass sie eine Adlige wäre die von '(ihrem) Papa, dem Lord of Papaver, geraubt (worden wäre)'\cite[S.255]{pir} Sich in Sicherheit wiegend greift der anwesende Admiral die Speckkugel an, während Mohnschnecke sich unter dem Vorwand in der Kapitänskajüte verstecken zu wollen und in von dort aus in den 

Eclair stößt nach seiner Verbannung aus dem Hause Mohn \cite[S.41]{pir} und einer Nacht unter einem Boot am Strand \cite[S.43]{pir} auf ein Pelmenidorf \cite[S.43]{pir} und lernt dort Otto kennen, dem er beim Fischen helfen will, indem er ihn aufs Meer rudert.\cite[S.45]{pir} So bezeichnet ihn der 'kleine Pelmen'\cite[S.45]{pir}, da Eclair 'nicht auf Geld aus ist (und) nicht ständig an die Zukunft (denkt) (...) einfach so (lebt), wie es gerade kommt' in Kapitel 8 als 'ganz gewöhnliche(n) Pelmen' \cite[S.68]{pir}. Später wird Eclair für kurze Zeit ins Pelmenidorf zurückkehren \cite[S.515ff]{pir}Als Otto von seinem Traum erzählt, dass er 'ein wilder Pirogg sein (möchte)'\cite[S.69f]{pir} und '(er) sogar ein eigenes Schiff (habe)' \cite[S.70]{pir} begutachten sie zunächst die beschädigte Brigantine\cite[S.75f]{pir}. Diese reparieren sie fortan abends, nach dem Fischen auf dem Meer\cite[S.90]{pir} in der Höhle, in der Otto sie versteckt hatte \cite[S.74f]{pir}. Das Schiff, das sie Feuerpfeil getauft haben\cite[S.92f]{pir} statten sie trotz aufgebrauchter finanzieller Ressourcen - '(s)ämtliche Ersparnisse des kleine Pelmen sowie der Erlös aus dem Verkauf des alltäglichen Fangs waren aufgebraucht'\cite[S.105]{pir}- durch eine List Eclairs mit dem letzten wichtigen Schiffszubehör aus\cite[S.106ff]{pir} und stechen in See \cite[S.136]{pir}, wo sie erst einmal vor Piroggen flüchten und sich verstecken müssen \cite[S.153ff]{pir} Allerdings geraten sie dort in einen Sturm, '(d)as Schiff sinkt' \cite[S.204]{pir} und die beiden Freunde '(schaukeln)(g)emeinsam mit dem Mast (...)auf den Wellen'\cite[S.205]{pir} Von Reispiroggen werden sie 'aus (dem) Meer gezogen'\cite[S.213]{pir} und reisen sie nun mit ihnen, helfen auf dem Schiff\cite[S.217]{pir} und überfallen gemeinsam Schiffe\cite[S.279ff]{pir}, versenken ein mit Grütze geladenes Wurstschiff,\cite[S.315]{pir} durch Eclairs Einfall mit einer Pumpe einen Wasserstrahl darauf zu richten \cite[S.310ff]{pir} und bergen mit Ottos Hilfe die dabei versunkene Schatztruhe.\cite[S.315f]{pir} Als die Pilzpiroggen die Besatzung der Reispiroggendschunke über die bevorstehende Schlacht bei Kap Trüffelgar informieren, segeln sie gemeinsam dort hin.\cite[S.318ff]{pir}
Nach der Schlacht bei Trüffelgar, weist Mohnschnecke Käptn Li an die Beiden 'so nahe wie möglich bei Murseille an Land gehen (zu lassen)'\cite[S.464]{pir} Sie reisen nach Piroggien \cite[S.479]{pir} wo es ihnen zunächst misslingt Otto durch den Piroggenbackofen zu schicken\cite[S.488]{pir}. Erst im Ur-piroggenbackofen \cite[S.494]{pir} gelingt es dem Pelmen seinen Traum zu verwirklichen\cite[S.497ff]{pir} Ihre Wege trennen sich danach \cite[S.504f]{pir}, Eclair kehrt ins Pelmenidorf zurück \cite[S.515]{pir} und Otto schließt sich der Speckkugelmannschaft an.\cite[S.523ff]{pir}Als Mohnschnecke verhaftet wird \cite[S.550]{pir} bittet Otto seinen Freund um Hilfe \cite[S.555]{pir}, zusammen begeben sie sich nach Murseille, wo sie verkleidet als reiche Ausländer\cite[S.565]{pir} zu Mohnschneckes Gerichtsverhandlung eingeladen werden\cite[S.568]{pir} und bei dieser Gelegenheit 'noch das Gerichtsgebäude inspizieren'\cite[S.577]{pir}um auszukundschaften, wie sie das Mädchen befreien können.\cite[S.569ff]{pir} Eclair hat eine Idee\cite[S.580]{pir} und sie planen\cite[S.581]{pir}und organisieren 'in reger Geschäftigkeit'\cite[S.584]{pir} um diesen umzusetzen. Bei ihrem Versuch Mohnschnecke verkleidet als Hochzeitsgesellschaft zu befreien\cite[S.584ff]{pir} geraten sie in einen Hinterhalt\cite[S.593]{pir} versuchen sie auf dem Richtplatz erneut zu befreien\cite[S.607ff]{pir} Nachdem Mordan verkleidet als Mönch, Mohnschnecke rettet, \cite[S.614]{pir} verlassen er und das Mädchen gemeinsam die Speckkugelmannschaft\cite[S.639]{pir} und macht Eclair ihr einen Heiratsantrag. \cite[S.640]{pir} Otto bleibt bei den Speckpiroggen \cite[S.638]{pir} und '(bringt) zufällig einen lange erloschenen Vulkan zum Ausbruch, den die Maisfladenstämme dieser Pelmencatépetl nannten'.\cite[S.646]{pir}

So verbringt Hörnchen nach seinem Sturz von der Speckkugel, bei der Entführung Mohnschneckes \cite[S.34]{pir}, zunächst mehrere Tage zu Hause im Bett. \cite[7. Kapitel]{pir}\cite[S.82]{pir} Nach seiner Reise mit Zwieback in der Obhut der Knofikarawane,\cite[S.124ff]{pir}bei der sie in Käsien einreisen\cite[S.140]{pir}die Hauptstadt Tschiesburg erreichen\cite[S.164ff]{pir} und auf dem Weg nach 'Njamburg, der Hauptstadt Wursterreichs'\cite[S.218]{pir} aus dem Hinterhalt von Blutwürsten angegriffen werden\cite[S.224ff]{pir}kommen sie auf die Burg des Vicomte de Brinson.\cite[S.233]{pir} Von dort fliehen sie\cite[S.239ff]{pir} wegen eines Missverständnisses mit Alalie, einer Käsewurst\cite[S.230f]{pir} und ihrem Vater.\cite[S.233ff]{pir} Nachdem sie, um der Willkür der wursterreichischen Grenzposten zu entgehen, \cite[S.285]{pir} in den von Alalie erhaltenen Wursthäuten\cite[S.242]{pir} einreisen \cite[S.287f]{pir}fliegt ihre Tarnung auf \cite[S.293]{pir} und sie werden in Njamburg\cite[S.289ff]{pir} als Spione verhaftet. \cite[S.293]{pir} Sie verbringen 'mehrere Tage (im Gefängnis), aber (...) niemand (kommt) um Hörnchen und Zwieback abzuholen.'\cite[S.322]{pir} Vor der Strafe die das Gericht, mit dem Herzog von Njamburg\cite[S.324ff]{pir} verhängt bewahrt sie zunächst der Knofiälteste \cite[S.327]{pir}, wegen des Einspruchs durch den Herzog von Brinson \cite[S.332]{pir} kommt es dann aber zu einem Turnier \cite[S.335ff]{pir} das auf 'dem Turnierplatz (...) vor dem Stadttor neben der Hauptstraße auf einer grünen Wiese'\cite[S.343]{pir}stattfindet. Als er dieses gewinnt \cite[S.357]{pir}, kehrt er nach dem Kauf einer vermeintlichen Mohnschnecke in Djadida \cite[S.388ff]{pir} 'gemeinsam mit derselben Knofikarawane, mit der sie schon einmal Käsien durchquert hatten' \cite[S.506]{pir} nach Murseille zurück \cite[S.506]{pir} wo er mit Halbpirogg später Rache plant \cite[S.540ff]{pir} und bis zum Ende der Geschichte bleibt: Hörnchen ist bei der Verhaftung Mohnschneckes auf Graf Napoleons Frühlingsball anwesend \cite[S.549ff]{pir}, im Gerichtssaal bei Mohnschneckes Verhandlung '(zieht) sich Eclair seinen Dreispitz tiefer ins Gesicht' \cite[S.570]{pir} um sich vor  ihm zu verstecken, duelliert sich mit Hörnchen,\cite[S.594]{pir} ist mit Zimtschnecke auf dem Richtplatz anwesend\cite[S.609]{pir} und heiratet schließlich Zimtschnecke. \cite[S.644]{pir}

Mordan, der Kapitän der Speckkugel unterbricht seine Zeit an Bord, als er freiwillig ins Kloster geht\cite[S.370]{pir} Im Spinatkloster wird er aufgenommen obwohl er 'gar kein Spinat (ist)(...)(und) nicht einmal ein Spargel oder Kohlrabi'\cite[S.401]{pir}, da er dem Abt eine großzügige Spende anbietet.\cite[S.402ff]{pir} Dort wird er unter anderem für seine Kritik am Essen im Kloster \cite[S.467ff]{pir} damit bestraft, dass er 'mit ausgebreiteten Armen auf dem Boden der Kapelle zu liegen und mit seinem einzigen Auge auf ein und dieselbe ausgetretene Fliese zu starren (hatte)'\cite[S.467]{pir} Als er in den frühen Morgenstunden Hunger bekommt  und sein Magen immer schlimmer knurrt, hofft er, dass alle im Kloster schlafen \cite[S.469]{pir} und er sich unbemerkt in die Klosterküche und in die Speisekammer schleichen kann. \cite[S.470f]{pir} Überwältigt von der 'Schatzkammer'\cite[S.470]{pir} an Essen die er in der 'Vorratskammer des Abts'\cite[S.470]{pir} findet, schenkt er sich '(f)ür den Anfang (...) einen Krug goldfarbenen, duftenden Rum aus dem Fässchen ein (...)(und) hielt ein langes und gründliches Gelage ab.'\cite[S.471]{pir} Als ihn die Mönche lauthals singend finden, \cite[S.471]{pir}beschimpft er diese erst mit 'Pfeffer und Pfanne! Der Salat greift an!' \cite[S.471]{pir}, bewirft sie mit Essen \cite[S.471]{pir} und wird von ihnen unter Protest und Beschimpfungen in seine Zelle gebracht.\cite[S.471]{pir} Als er wieder aufwacht ist er in seiner Zelle eingemauert um zu fasten und zu bereuen.\cite[S.472]{pir} Seine Taten schon bereuend, aber verzweifelt durch die Ungewissheit, was in der Welt ausserhalb des Klosters geschieht, packt er den Arm, der ihm einen Wasserkrug durch das Durchreichloch streckt und bedroht den Spinat, zu dem er gehört und befiehlt ihm zu sagen 'was in der Welt so los ist' \cite[S.583]{pir} und wann er hier rauskommt.\cite[S.583]{pir} Er antwortet mit der Nachricht, dass es heißt 'die Herrschaft der wilden Piroggen (habe) bald ein Ende'\cite[S.583]{pir} Er ist fassungslos und will das nicht zulassen, er kann 'einfach durch (die) Wand hindurch (gehen)'\cite[S.620]{pir}, weil 'Spinatmönche (...) keine Maurermeister (sind)' \cite[S.620]{pir} besucht vor Mohnschneckes Hinrichtung '(den) Spinatmönch, der bei der Urteilsvollstreckung den geistlichen Beistandleistet, einen Besuch in seiner Zelle ab, sperrte ihn in den Kleiderschrank' \cite[S.620]{pir}und taucht als dieser auf dem Richtplatz auf um sie zusammen mit den Speckpiroggen aus der Gefangenschaft zu befreien. \cite[S.614]{pir} Zurück auf der Speckkugel wird er wieder zum Kapitän \cite[S.634]{pir} und entdeckt mit den Speckpiroggen 'einen neuen Kontinent, dem er den rätselhaften Namen Amoreka (gibt) (...), wo er 'sich mit dem dortigen König an(freundet)'\cite[S.646]{pir} und auf einem von ihm geschenkten Küstengebiet 'eine Piroggenkolonie (gründet) und zu deren erste(n) Gouverneur ernannt. (wird'\cite[S.646]{pir}


Mohnschnecke Speckkugel
Mohnschnecke Gefängnis

\subsubsection{Wichtige Schauplätze}
Turnier
Trüffelgar	

\subsubsection{Andere geographische Angaben}
Im Laufe der Geschichte erfährt man auch von Orten, an denen entweder keine oder wenig Handlung stattfindet, die aber dennoch zur Welt gehören, von der erzählt wird.


	
\section{Strukturanalyse}
In der Strukturanalyse werden Ablauf, Spannungskurve und Erzählkurve behandelt.

\subsection{Ablauf}
Im Unterpunkt Ablauf wird auf die Komposition der Geschichte, mit ihrer Kern- und Rahmenhandlung eingegangen.

\subsubsection{Komposition der Geschichte}

„Die wilden Piroggenpiraten“ ist unterteilt in einen Prolog, 67 Kapitel und den Epilog. Im Prolog wird der Leser in die Geschichte und den Kosmos der Piroggen eingeführt. Die Handlung lässt sich in vier Teile unterteilen, die durch drei Ereignisse abgetrennt werden: das Übergangskapitel 4, nach dem Eclair und Hörnchen getrennte Erzählstränge – also auch wie Mohnschnecke eigene Kapitel haben – die Schlacht von Trüffelgar (Kapitel 41 – 44) und Mohnschneckes Gefangennahme (Kapitel 55- 65) . Der erste Teil (Kapitel 1-3) handelt von Mohnschnecke, der Hauptprotagonistin, Eclair und Hörnchen. Am Ende des dritten Kapitels wird Mohnschnecke entführt. Im zweiten Teil (Kapitel 5-42) machen sich Eclair, begleitet von Otto (Kapitel 13), und Hörnchen (Kapitel 14) auf den Weg Mohnschnecke zu suchen. Bis auf einen kurzen Berührpunkt der Erzählstränge in Kapitel 22, gehen alle drei Figuren getrennte Wege und haben eigene Kapitel. Vor und bei der Schlacht von Trüffelgar sind Eclair und Mohnschnecke in denselben Kapiteln zu treffen, da sie sich in unmittelbarer Nähe befinden und sich schließlich auf er Speckkugel treffen,  Mohnschnecke schickt Eclair und Otto aber zurück nach Murseille, sodass die Protagonisten wieder in abgegrenzten kapiteln zu finden sind. Hörnchen bringt währenddessen eine verschleierte Zimtschnecke nach Murseille, weil er sie für Mohnschnecke gehalten hat. Im dritten Teil entschließt sich Mohnschnecke auf Graf Napoleons Ball zu fahren, wo sie verhaftet wird. Die Erzählstränge überkreuzen sich hier häufiger, in Kapitel 53 kommt Mordan im selben Kapitel vor wie Mohnschnecke und Halbpirogg, obwohl er im Kloster ist, im Darauffolgenden schmieden Hörnchen und Halbpirogg zusammen einen Plan, den sie in Kapitel 55 ausführen und Mohnschnecke gefangen nehmen lassen. In Kapitel 59 sehen sich das Mädchen und Eclair im Gerichtssaal. 

\subsection{Kernhandlung}

In der Geschichte dreht sich alles um das Mädchen Mohnschnecke. Sie wohnt in Murseille und möchte auf einen Ball nach Sankt Krokant, wo sie bei einem Yachtausflug von Piroggen entführt wird. (Kapitel 1-3) Mordan, der Kapitän verliebt sich in sie und führt wegen ihr einige Neuerungen ein, die der Mannschaft zunächst nicht gefallen. (Kapitel 15) Durch ihr selbstbewusstes Auftreten, ihren Ehrgeiz in Kampfangelegenheiten und ihre Intelligenz, mit der sie den Seeräubern mehrere Male hilft, lebt sie sich schnell auf der Speckkugel ein. Als ihr der Kapitän einen Heiratsantrag macht lehnt sie dankend ab, woraufhin Morgan das Schiff verlässt und Mohnschnecke beginnt nun ihr eigenes Denken und Handeln zu hinterfragen. Bei der Schlacht von Trüffelgar bei der der Großteil der weltweiten Piroggen gegen die Spunier kämpft, setzt sich das Mädchen als Anführerin durch. Eclair, dem sie danach begegnet und der sie gern nach Hause bringen möchte, weist sie ab und schickt ihn weg. Dennoch lässt sie der Gedanke an den Ball auf Sankt Krokant nicht los, sodass sie mit der Speckkugel bis kurz nach Sankt Krokant segelt. Verkleidet auf dem Ball um nicht erkannt zu werden, wird sie festgenommen: Halbpirogg und Hörnchen haben sie verraten. Eclair und die Speckpiroggen versuchen mit allen Mitteln sie zu befreien.

\subsection{Rahmenhandlung}

Die Rahmenhandlung bilden Eclair und Hörnchen.

Zunächst rudert Eclair mit Hörnchen nach Mohnschneckes Entführung zurück nach Hause. Im Hafen, wo Hörnchen von seinem angeblichen Kampf berichtet, kann der geschockte Eclair Mohnschneckes Vater kaum widersprechen und stößt, nachdem er in Mohnschen Haus nicht mehr willkommen ist und kein Geld mehr hat, nach einer Nacht am Strand auf eine Pelmenisiedlung. Dort lernt er Otto kennen, mit dem er eine Brigantine repariert, um mit ihn zu einem Piroggenbackofen zu begleiten und nach Mohnschnecke zu suchen. Als die beiden nach einem Sturm schiffbrüchig sind, werden sie von Reispiroggen gerettet, mit denen sie zur Schlacht von Trüffelgar fahren. Nachdem Mohnschnecke sie dort nach einem Sieg entlohnt und nach Hause schickt, begleitet Otto seinen Freund zunächst in einen Piroggentempel und, nachdem das Backen dort aufgrund aufgebrachter Piroggen scheitert, auf einen Vulkan, wo Otto der Pelmen sich schließlich seinen Traum erfüllt. Zunächst gehen die Beiden nun getrennte Wege, Otto schließt sich den Piroggen auf der Speckkugel an und Eclair kehrt zurück ins Pelmenidorf, aber als Mohnschnecke gefangen genommen wird will Eclair seinem Freund helfen sie zu befreien.

Hörnchen liegt zunächst faul zu Hause und bricht erst zu einem Befreiungsversuch auf, als ihn sein Vater losschickt. Er schließt sich zusammen mit Zwieback, seinem Begleiter, einer Knofikarawane an. Nach einer Schlacht mit Blutwürsten denkt die gerettete Käsewurst Alalie, dass Hörnchen sie zur Frau nehmen will, sodass ihr Vater bei ihrer Ankunft eine Verlobungsfeier vorbereitet. Da dies nie seine Absicht war, flieht Hörnchen mit Zwieback und wird von Alalie verabschiedet, die ihnen noch Blutwursthäute als Verkleidung mitgibt. Mit diesen überschreiten sie die Grenze nach Wursterreich, wo sie als Spione auffliegen und verhaftet werden. Bei ihrer Verhandlung werden sie mit Hilfe des Knofiältesten zwar freigesprochen, der Einspruch des Vicomte de Brinson führt jedoch dazu, dass sich er und der Vicomte duellieren müssen. Nach einem Sieg fahren Hörnchen und Zwieback auf der Straße von Djadida auf den Sklavenmarkt und kaufen dort ein vermummtes Mädchen, das sie für Mohnschnecke halten. Zurück zu Hause stellt sich herraus, dass es sich um eine Zimtschnecke handelt. Da Hörnchen nun das Gespött der Stadt ist, ist er umso empfänglicher für Halbpiroggs Racheplan Mohnschnecke beim Frühlingsball verhaften zu lassen.


\section{Spannungskurve}

Das Buch enthält viele verschiedene Spannungskurven die sich teilweise auf einzelne Personen aber auch Personengruppen beziehen. Hier werden die Höhepunkte und die Wendepunkte der Geschichte aufgezeigt.

\subsection{Höhepunkte}
Eine Spannungskurve erstreckt sich vom Kapitel eins bis drei, bis zur Entführung Mohnschneckes. Die erste Auseinandersetzung mit den Spuniern (Kapitel 26) schauckelt sich über Kapitel 27 und 28 hoch, flacht dann wieder ab, bis im Kapitel 39 die Nachricht von den Pilzpiroggen über die Schlacht von Trüffelgar überbracht wird, die dann der Höhepunkt ist. Eine Kurve, die sich langsam aufbaut ist die von Halbpirogg: ab Kapitel 18 kommt immer wieder sein Unmut über Mohnschnecke zum Ausdruck, der von seiner Niederlage im Kampf um den Kapitänsposten und die Ablehnung durch die Mannschaft wegen seines schlechten Essens,  genährt wird. Er findet seinen Höhepunkt in seiner Rache beim Verrat auf dem Frühlingsball und dem Versuch der Übernahme der Speckkugel . Ab Mohnschneckes Festnahme steigt auch hier eine Spannungskurve steil an, die Ereignisse überschlagen sich und der Erzählstrang wird immer mehr wieder zu einem zusammengefügt, die Ansammlung immer mehr, vorher unabhängig erzählter Erzählperspektiven häufen sich. Kleinere Spannungskurven erstrecken sich über einzelne Kapitel. Zum Beispiel Mohnschneckes erste Schlacht, der Angriff auf Mordan, den sie abwehrt, Mordans Heiratsantrag und als sie verkleidet auf den Frühlingsball geht ; Eclairs List, mit der er und Otto zum letzten fehlenden Schiffzubehör kommen, das Kapitel in dem sie vom Reispiroggenschiff gerettet werden und als Otto zunächst versucht im Piroggentempel und dann auf dem Vulkan gebacken zu werden. Hörnchens zunächst eher unspektakulärer Weg, wird mit der Blutwurstschlacht spannender. Die darauffolgende vermeintliche Verlobung, die zuerst in Vergessenheit gerät führt nach der Festnahme in Wursterreich, bei der Gerichtsverhandlung zu einem Höhepunkt: dem Turnier. Nachdem er dieses als Sieger verlässt und eine vermummtes Mädchen kauft, wird es bis zur Enthüllung in Murseille spannend.

\subsubsection{Wendepunkte}

Die Entführung Mohnschneckes setzt das von Eclair und Hörnchen angestrebte Ziel der Geschichte: ihre Befreiung.  Die Veränderungen an Bord, die durch sie indiziert werden, führen zu einem anderen Verhalten bei Halbpirogg, (Kapitel 18) der durch seinen Unmut die Geschichte mit seinem Verrat später entscheidend beeinflusst. Mordans Heiratsantrag, den Mohnschnecke ablehnt, bildet die Grundlage für ihre Beförderung zum Käptn und einer Veränderung ihrer Stimmung. 

Zunächst scheint es als hätte Eclair aufgegeben, doch dann erzählt ihm Otto im Pelmenidorf von der Brigantine und von seinem Wunsch ein Pirogg zu werden. Als die Beiden nach einem Sturm hoffnungslos im Meer treiben, werden von Reispiroggen auf ihrem Schiff aufgenommen und fahren mit zur Schlacht von Trüffelgar. Eigentlich scheint Eclairs Ziel erreicht, als er Mohnschnecke nach der Schlacht auf der Speckkugel begegnet, doch für ihn unerwartet, will das Mädchen nicht mit nach Hause kommen. Er geht zurück ins Pelmenidorf, wo er später von Otto, der, nachdem er mit Eclair auf einem Vulkan war, ein Pirogg ist um Hilfe bei der Befreiung Mohnschneckes.

Erst wegen seines Vaters fängt Hörnchen an nach Mohnschnecke zu suchen. Die Verkleidung, die er von Alalie bekommt, ist zwar bei der Grenzüberschreitung nach Wursterreich von Nutzen, aber führt auch zur Festnahme als Spion und bei der Verhandlung zum Turnier. In Djadida glaubt er die vermummte Mohschnecke gefunden zu haben, zu Hause stellt sich herraus, dass er hinters Licht geführt wurde. 
Entscheidend für das Happy End ist auch, dass die Nachricht über Mohnschneckes Verhaftung Mordan im Kloster erreicht, sodass er sie überraschend am Tag ihrer Hinrichtung retten kann.


\section{Erzählung}
Im Folgenden werden Erzählperspektive, Erzähltechnik und Erzählzeit aufgeführt.
\subsection{Erzählperspektive}

Die Geschichte wird sowohl auktorial, als auch personal widergegeben. Der Erzähler hat einen Überblick über das Gesamtgeschehen und berichtet  beobachtend sowohl über Vergangenheit, als auch Gegenwart, was im Prolog, im Epilog, als auch in der Geschichte selbst geschieht. 
Im Prolog wird der Leser zunächst in die Welt der Piroggen eingeführt und lernt die Stadt Murseille kennen. Während des Handlungsverlaufs, bei denen die Erzählstränge der einzelnen Protagonisten parallel ablaufen, kann der Erzähler sowohl aus der Sicht einer oder mehrerer Personen vom aktuellen Geschehen berichten, als auch auf vergangenen Ereignisse zurückgreifen und vermittelt dadurch die Welt der Wirklichkeit der Protagonisten. Zum Beispiel kennt er die Vergangenheit und den Hintergrund Graf Napoleons (S.17f), erklärt die Eigenschaften der Bewohner von Käsien (S.140f) und kennt die Intentionen, Gedanken und Gefühle Mohnschneckes als Mordan das Schiff verlässt (S.374). Als Mohnschnecke entführt wird (Kapitel 3) die Speckkugel an Eclair und Otto vorbeifährt (Kapitel 22) und bei der Gerichtsverhandlung (Kapitel 59) wird gleichzeitig aus mehreren Perspektiven erzählt.



\subsection{Erzähltechnik}

Es sind verschiedene Erzählweisen zu finden. Meist wird berichtend erzählt, der Erzähler beschränkt sich auf Beschreibungen und berichtet dabei  subjektiv, etwa über die Geschehnisse an Bord und als sich „Halbpirogg (…) wieder einen hinter die Binde gegossen (hatte)“ (S.298) und bei der Schlacht von Trüffelgar, „als in der Meerenge (…) wahrlich merkwürdige Dinge (geschahen)“ (S.444). Szenisches Erzählen wird verwendet, als im Hafen auf die Brigantine und die Einlösung der Wettschuld gewartet wird (S.117), der Dialog zwischen Hörnchen und Alalie vor dem Turnier (S.340) und die Unterhaltung zwischen Eclair und Otto vor der Schlacht bei Trüffelgar (S.414ff) Als Eclair und Otto schiffbrüchig sind, werden seine Gefühle in Erlebter Rede zum Ausdruck gebracht (S.209f, Zitat 1), ebenso als Mordan das Schiff verlässt und Mohnschnecke die Verantwortung überlässt,(S.375, Zitat 2).


„Er fühlte sich völlig aufgeweicht.“ „Er wollte gar nichts. In seinem Körper hauste keinerlei Verlangen mehr, höchstens der Wunsch, einzuschlafen und nie mehr aufzuwachen.“
„Verfluchter Mordan“ Befördert sie auf den Kapitänsposten ohne sie überhaupt zu fragen, ob sie das möchte!“

\subsubsection{Erzählzeit}

Im Laufe der Handlung kommt es sowohl zu Zeitraffungen, als auch zu Zeitdeckungen in der Erzählzeit. In den einzelnen Kapiteln ist generell eine Zeitdeckung zu finden, die erzählte Zeit und die Erzählzeit sind in etwa gleich lang. Bei der Entführung Mohnschneckes in Kapitel 3 und bei „Admiral Morcillas letztes Gefecht“ (kapitel 44.) wird zeitgleich aus mehreren Perspektiven erzählt. Ansonsten kommt es meist zu Zeitraffungen, da die Erlebnisse der einzelnen Hauptprotagonisten abwechselnd erzählt werden.  Nach dem Berührpunkt mit Eclair in Kapitel 22, erzählen bis zu Schlacht von Trüffelgar 6 Kapitel von Mohnschnecke und lediglich 3 von Eclair.

\section{Sprachanalyse}
 Bei der Sprachanalyse werden semantische und stilistische Besonderheiten aufgezeigt und zitiert.

\subsection{Semantik}

Semantische Analyse
Hierbei wird das Augenmerk auf die einzelnen Wortarten gelegt, d. h. es werden die sogenannten Inhaltswörter hinsichtlich deren Häufigkeit im untersuchten Text und ihren semantischen Gehalt hin untersucht. 

\subsubsection{Lebensmittel}

Die am häufigsten verwendeten Wörter lassen sich unter dem Hyperonym Lebensmittel zusammenfassen. Sie sind aus dem alltäglichen Gebrauch bekannt, aber speziell auf das Bäcker- und Metzgerhandwerk und allgemein auf die Lebensmittelindustrie zurückführen. Die Art der Substantive gibt im Zusammenhang mit den dabei verwendeten Lebensmittelnamen Aufschluss über die Funktion des verwendeten Namens. 
\\
So sind die Eigennamen der Nahrungsmittel ebenso Eigennamen von Personen in der Geschichte. Von den Hauptprotagonisten Mohnschnecke,\cite[S.13ff]{pir} Hörnchen,\cite[S.25ff]{pir} Eclair,\cite[S.15ff]{pir} über die Nebendarsteller, Otto den Pelmen (Pelmeni),\cite[S.45ff]{pir} Halbpirogg (Pirogge) und die restlichen Piroggen,\cite[S.33ff]{pir} Mohnschneckes Eltern Mohnpotitze \cite[S.13]{pir} und Mohnstrudel \cite[S.13]{pir}, Hörnchens Vater Laib \cite[S.40]{pir} und seine Mutter Baguette \cite[S.63]{pir}, sowie den Arzt Doktor Korn \cite[S.39]{pir} und zum Beispiel alle im Hafen anwesenden Gebäckstücke 'Plunderstücke unterschiedlicher Füllung, (...) Kümmelstangen, Zwiebäcke. Biskuits, knusprige Waffeln und cremegefüllt Waffelröllchen, Krapfen, Bagel (...) Maisflips (usw.)'\cite[S.38]{pir} und 'Bürgermeister Kameruner (...) und Eierkuchen, seine Frau',\cite[S.189]{pir} handelt es sich ausnahmslos um Lebensmittel. Neben Gebäck sind im Rest der Geschichte aber auch noch Andere vertreten. Hörnchen begegnet zum Beispiel in einer käsischen Pizzeria einer Pizza,\cite[S.171]{pir}, der Käsewurst Alalie mit ihren Zofen, zwei Mozzarellas,\cite[S.179]{pir} Blutwürsten \cite[S.220ff]{pir}, einem Kartoffelknödel, \cite[S.223]{pir}, dem Erzbischof Kürbis, \cite[S.289]{pir}, einem Kohlrabi \cite[S.289]{pir} und Mordan im Spinatkloster einem Spinatnovizen \cite[S.400f]{pir} und einem Spargel.\cite[S.401]{pir} Ihre Fortbewegungsmittel sind Maiskolbenpferde, \cite[S.124]{pir}\cite[S.241]{pir} Auberginenkamele,\cite[S.126]{pir} Zuccinipferde, \cite[S.286]{pir} 'zwiebellauchgeschmückte Gurkenpferde' \cite[S.289]{pir} und Melonenpferde.\cite[S.291]{pir}
\\
Vereinzelt bezeichnen Eigennamen auch geografische Angaben, etwa das Halvagebirge,\cite[S.165]{pir} Sankt Krokant,\cite[S.19ff]{pir} die Marzipankapelle \cite[S.18]{pir} die Mündung der Schweppes ins Meer,\cite[S.187]{pir}, die Frikadellenpier,\cite[S.21]{pir} Piroggien,\cite[S.479ff]{pir} Döner am Bratpfanngestade, \cite[S.381]{pir} Torte auf Tortilla,\cite[S.405]{pir} das Cheddarviertel \cite[S.169]{pir} und die zwölf Hügel von Tschiesburg: \cite[S.165]{pir} 'Parmesan, Cheddar, Camembert, Brie, Appenzeller, Backstein, Gorgonzola, Edamer, Brimsen, Mozzarella, Feta und Roquefort.\cite[S.165]{pir}
\\
Gattungsnamen beziehen sich auf Länder und Herkunftsorte, wie Käsien \cite[S.140ff]{pir} mit seiner Hauptstadt Tschiesburg,\cite[S.164ff]{pir} Wursterreich \cite[S.285ff]{pir} mit der Hauptstadt Njamburg, \cite[S.289]{pir} Austerreich, aus dem Mohnschneckes Mutter stammt, \cite[S.13]{pir} die Feingebäckprovinz,\cite[S.325]{pir} Grilland \cite[S.635]{pir} und das Bergkäsleinkloster. \cite[S.164]{pir}
\\
Sammelnamen beziehen sich, wie auch die Wortgattung, auf einen Zusammenschluss, hier von bestimmten Lebensmitteln in Familien. So stammt Hörnchen aus einer Weißbrotsippe,\cite[S.538]{pir} die die Kleiekuchen \cite[S.538]{pir} 'nicht ausstehen können'\cite[S.538]{pir} und 'eine Brimsenfamilie (war) im Jahrhundert der großen Völkerwanderung auf der Flucht vor den barbarischen Nomadenvölkern der Brezeln'.\cite[S.165]{pir}


\subsubsection{Nautische Termini}

Neben den Lebensmitteln sind Fachausdrücke aus der Nautik die am zweithäufigsten verwendeten Nomen.
Sie werden vor allem zusammen mit Mohnschnecke und Mordan auf der Speckkugel und mit Eclair und seinem Freund Otto verwendet, da in beiden Fällen auf dem Wasser gereist wird.
\\
Angefangen bei einfachen allgemein bekannten und auch im normalen Sprachgebrauch verwendeten Ausdrücken wie Pier,\cite[S.21]{pir} Navigation, \cite[S.138]{pir} Kompass,\cite[S.139]{pir} Ruderboot,\cite[S.21]{pir} Yacht,\cite[S.21]{pir} Gallionsfigur,\cite[S.30]{pir} Fahne, \cite[S.51]{pir} und Kapitän \cite[S.35]{pir} werden auch einschlägig nautische Termini verwendet.
\\
So ist Halbpirogg ein Kapitänsmaat,\cite[S.50]{pir} die Speckkugel verfügt über eine Reling, \cite[S.8]{pir} einen Schiffsrumpf,\cite[S.31]{pir} ein Deck,\cite[S.49]{pir} eine Kommandobrücke,\cite[S.53]{pir} einen Enterhacken,\cite[S.53]{pir} Bordwände,\cite[S.51]{pir} einen Klüver,\cite[S.51]{pir} eine Kajüte, \cite[S.55]{pir} Beiboote, \cite[S.56]{pir} ein Lateinersegel, \cite[S.56]{pir} einen Mast, \cite[S.56]{pir} eine Reling \cite[S.57]{pir}  und ein Krähennest. \cite[S.95]{pir} Auch liest man von Taljen, \cite[S.104]{pir} Tauen, \cite[S.104]{pir} Großsegeln \cite[S.105]{pir} 'Bändsel(n) (die) man gut zum Reffen der Segel gebrauchen (kann)',\cite[S.106]{pir} 'Großbäume(n), Gaffeln, Salings und andere(n) Ersatzteile(n) für Segelschiffe'. \cite[S.107]{pir}
\\
Es is die Rede von verschiedenen Booten und Schiffen, von Kriegsschiffen, \cite[S.7]{pir} Dschunken, \cite[S.8]{pir} 'schmalen, langen Ruderbooten', \cite[S.8]{pir} einer Brigg, \cite[S.35]{pir} einem Kaufmannsschiff, \cite[S.57]{pir} einem halbmilitärischen Kurierschiff, \cite[S.94]{pir} eine portugiesische Galleone, \cite[S.94]{pir} einer Brigantine, \cite[S.105]{pir} Fregatten, \cite[S.443]{pir} einer Galeotte, \cite[S.406]{pir} einer Karavelle, \cite[S.35]{pir} einer Schebecke, \cite[S.406]{pir} einer Schonerbark, \cite[S.406]{pir} von 'Plankenbooten(n), (...) (K)ajaks, (...) (D)rakkar(en) (...) und (...) eine(r) (...) (G)aleere.' \cite[S.407]{pir} Zudem kann man sich als Unwissender oder, wenn man nicht in der Nähe der See aufgewachsen nicht vorstellen mit was man es zu tun hat, wenn von Untiefen,\cite[S.7]{pir} Sandbänke(n),\cite[S.7]{pir}, Riffe(n),\cite[S.7]{pir} einer Flotte, \cite[S.8]{pir} der Gischt,\cite[S.51]{pir} von einem '(unverändert(en) Kurs',\cite[S.313]{pir} Kielwasser \cite[S.404]{pir} und dem Flaggenalphabet \cite[S.407]{pir} und im Zusammenhang mit Kanonen und Piraten von Lunten\cite[S.51]{pir} gesprochen wird.

Zu ein paar ausgewählte Wörter sind unter 'Worterklärungen' erklärt.\cite[S.647]{pir}

\subsubsection{}

\subsection{   }
Art der Adjektive (beschreibend, bewertend),

\subsection{   }
Art der Verben (Handlungs-, Beschreibungs-, Modal- und Hilfsverben, passive und aktive Verben),

\subsection{   }
Ausdruckswert der Wörter (Denotation, Konnotation, übertragene Bedeutung).
Denotation Herz alle wissen Liebe, bewusst unbewusst mitschwingt, Konnotation Gegenteil
übertragene Bedeutung Feuer aus. Liebe erloschen



\subsection{Stilistik}

\subsubsection{Erzählstil}

\subsubsection{Aliteration}

\subsubsection{Personifikationen}

Alle im Buch genannten Charaktere sind Personifikationen von Lebensmitteln. Von den Hauptfiguren Mohnschnecke,\cite[S.13]{pir} Eclair \cite[S.15]{pir} und Hörnchen \cite[S.25]{pir} über den Quarkplunder, der bei den Mohns gearbeitet hat,\cite[S.15]{pir} die gesamte Speckpiroggenmannschaft,\cite[S.47ff]{pir} Maultasche,\cite[S.108]{pir} der Inhaber eines Segelgeschäfts,\cite[S.107f]{pir} zwei Pilzkrapfen,\cite[S.113]{pir} ein Hühnerfleischpelmen,\cite[S.116]{pir} der ‚Lieblingshund Rollmops‘,\cite[S.117]{pir} ‚(die) Kapitäne Sobrasada, Cantimpalo und Butifarra, allesamt weißlich angeschimmelte lange Würste mit goldenen Epauletten‘,\cite[S.253]{pir} ein ‚von oben bis unten mit einer struppigen Schnur umwickelter Räucherschinken mit einem braunen Al-Capone-Hut auf dem Kopf‘,\cite[S.291]{pir} (d)er Pilzpiroggenanführer,\cite[S.318]{pir} bis zu den Bananen,\cite[S.385]{pir} von denen ‚jeder wusste, was für schlechte Arbeiter (sie) waren‘.\cite[S.385]{pir}Auch die  ‚Cracker (…), die von Roquefort, Camembert, Gorgonzola und anderen Edelschimmelkäsesorten (abstammen sollten)‘,\cite[S.140]{pir} sind ebenso personifizierte Lebensmittel, wie der Hauptmann der Käsecrackerwache,\cite[S.143]{pir} die Pumpernickel in Madeira,\cite[S.195]{pir} eine Krakauer Wurst,\cite[S.285]{pir} ‚dicke Mettwürste‘,\cite[S.291]{pir} ‚bleiche Milchwürstchen‘,\cite[S.291]{pir} eine Salami,\cite[S.291]{pir} ‚Leberwürste, Fladenbrote, Eierkanapees und -kuchen, Bockwürste, und Knacker, leichtsinnige Pizzen und reich gekleidete Tschiesburger, bettelarme Kartoffelpuffer und pralle Rollschinken, narbige Piroggen und bröselige Käsecracker, hübsche und schlanke Krapfen und schüchterne Knödel‘,\cite[S.382]{pir} wie viele der Piraten die sich versammeln, Kaldaunenpiroggen,\cite[S.420]{pir} Reispiroggen \cite[S.420]{pir} und Kohlpiroggen,\cite[S.420]{pir}


\subsubsection{Allusionen}

Allusionen findet man im geschichtlichen und im geographischen Zusammenhang.

Zu Beginn der Geschichte erfährt man von Eclair, dessen ‚(l)ängere Studien an der Bonbonne (…) seine geistigen Fähigkeiten vervollkommnet (hatten)‘\cite[S.15]{pir} Er ist ‚(d)er letzte Sprössling eines während der Revolution ruinierten Adelsgeschlechts (…) war bettelarm, sonst hätte er nicht sein Heimatland verlassen und Arbeit als Verkäufer in einem Mohngeschäft suchen müssen.‘\cite[S.15f]{pir} Die Ähnlichkeit des Wortklangs von Bonbonne und Sorbonne, sowie die Revolution, die sein Adelsgeschlecht ruiniert hat, lassen darauf schließen, dass es sich um eine Andeutung zur französischen Revolution handelt und die Hauptstadt, des uns bekannten Frankreichs gemeint ist. Auch der Gleichklang mit der Ausgangsstadt ‚Murseille‘\cite[S.13]{pir} und der französischen Mittelmeerstadt Marseille unterstützen die Möglichkeit eines Zusammenhangs. Ebenso wie die Informationen die über ‚Graf Napoleon‘ \cite[S.17f]{pir} gegeben werden.
‚Graf Napoleon von und zu Schnitte - ein Schichtkuchen mit Sahnefüllung - (der) (…) eine überaus prominente Persönlichkeit in der Stadt (Murseille) (ist)‘\cite[S.17]{pir} und der ‚versuchte, ins Königreich Kästen einzumarschieren - mit der Absicht es zu erobern und sich als Kaiser einzusetzen.‘\cite[S.18]{pir} Wie der Name schon sagt, handelt es sich um Napoleon Bonaparte, dessen Arme beim Russlandfeldzug stark geschwächt wurde‚ so ‚geriet (sie) weder groß noch ausreichend kriegerisch‘\cite[S.18]{pir} und der Versuch in das Königreich Kästen einzumarschieren \cite[S.18]{pir} endete (…) mit einem unbedeutenden Scharmützel im Grenzgebiet.‘ Hier wird auf die Schlacht von Waterloo angespielt, die bei der belgischen Stadt Waterloo stattfand und damals zu den Niederlanden gehörte. Dementsprechend kann man daraus schließen, dass mit Kästen die Niederlande gemeint sind. ‚Eine Eskorte englischer Puddings brachte ihn nach Sankt Krokant.‘\cite[S.18]{pir} Dabei ist die Verbannung Napoleons auf St. Helena gemeint. Der einzige Sachverhalt der stören könnte ist die Beschreibung ‚Die Insel lag in nicht allzu großer Entfernung vor Murseille in der Meeresbucht, gehörte aber offiziell nicht mehr zum Stadtgebiet. Also scheint es sich um eine Vermischung zwischen der Verbannung Napoleons nach St. Helena, also St. Krokant und die Nähe zu Murseille, also seinem ersten Verbannungsort Elba zu handeln, der aber ebensowenig nah an der Ausgangsstadt liegt, aber es handelt sich auch um eine fiktive Welt, sodass die Relationen nicht immer genau sein müssen.

Geographische Angaben sind sind etwa die Stadt Kästen, \cite[S.18]{pir} mit der Hauptstadt Tschiesburg.\cite[S.164]{pir} Über die Reise Hörnchens über die ‚Grafschaft Cheddar‘,\cite[S.164]{pir} ‚(die) scharf riechenden Dörfer des Baron Gorgonzola‘, das ‚Bergkäslein-Kloster‘,\cite[S.164]{pir} ‚das phantastische weiße Schloss Parmesanstein \cite[S.164]{pir} und Tschiesburg mit seinen ‚massiven gelben Mauern‘,\cite[S.164]{pir} kann man nur spekulieren. Zumindest Parmesanstein könnte an das Schloss Neuschwanstein angelehnt worden sein. Der Scirocco \cite[S.66]{pir} lässt darauf schließen, dass sich Eclair bei den Pelmeni in Italien aufhält, nur dort wird dieser Wind, der von der Sahara über das Mittelmeer weht, so genannt. Die Stadt Venezia entspricht unserem Venedig, ein Schiff aus Venezia hat ein Banner bei sich,\cite[S.51]{pir} das ein(en) goldene(n) Löwe(n) (zeigt), der sich auf die Hintertatzen erhoben (hat), auf dunkelrotem Grund.‘\cite[S.51]{pir} Djadida [S.50]{pir} und die Straße von Djadida [S.359]{pir} sind die Pendants zu Gibraltar und der Straße von Gibraltar.

\subsubsection{Euphemismen}

Euphemismen können zur Aufwertung, zur Milderung und Schonung und zur Tarnung tatsächlicher Vorgänge verwendet werden.
Aus Höflichkeit und Wertschätzung nennt Käptn Mordan Mohnschnecke nach dem Angriff auf ein venezianisches Handelsschiff \cite[S.313]{pir} 'mein kleines Feingebäck.'\cite[S.61]{pir}

Am Meisten werden Euphemismen im Zusammenhang mit abgeschwächten Flüchen verwendet und im umgangssprachlichen Gebrauch benutzte Kraft- und Vulgärausdrücke umgewandelt. Hier wird erneut der Zusammenhang mit Lebensmitteln aufgenommen. So liest man beim Angriff der Piroggen auf die Zuckerstern \cite[S.30ff]{pir}'Zerbröseln! Zerbröseln',\cite[S.33]{pir} als sie überlegen wie sie Hörnchen umbringen können\cite[S.33]{pir} sie lachen über ihn 'Seht mal, er hat die Hosen voll Teig',\cite[S.33]{pir} weil er sich vor Angst in die Hosen macht \cite[S.33]{pir} und es heißt 'zurück mit dem Knödel',\cite[S.34]{pir} als sie erfahren, dass er wohlhabende Eltern hat.\cite[S.34]{pir} Käptn Mordan ruft aus Freude und Überraschung 'Speck und Pfeffer!',\cite[S.51]{pir} als er ein venezianisches Kaufmannsschiff erkennt \cite[S.51]{pir} und fügt hinzu 'dass euch der Speck ranzig wird und die Seite schimmelt, wenn irgendetwas schiefgeht!'\cite[S.51]{pir} um der Mannschaft zu verdeutlichen, dass sie sich anzustrengen haben. Halbpirogg findet, dass Mohnschneckes Einfluss, den Käptn 'zu einem richtigen Vollknödel (macht)',\cite[S.145]{pir} im normalen Sprachgebrauch wäre wohl der Ausdruck 'Vollidiot' gefallen. Als Käptn Mordan den Angriff auf Madeira startet, befiehlt er 'diesen Mehlbeeren einen ordentlichen Schrecken ein(zu)jagen'\cite[S.188]{pir}, so wird der drastische Ausdruck Feigling umgangen. Beim Ausbruch Hörnchens und Zwiebacks aus der Burg des Vicomte de Brinson, \cite[S.240]{pir} befürchtet Hörnchen aus dem Fenster zu fallen und zu 'zerbröseln',\cite[S.240]{pir} statt zu sterben oder zu zerschellen. In der Verkleidung als Würste \cite[S.286]{pir} meint das Weißbrot zu seinem Gefährten 'Du siehst vielleicht aus... (...) wie ein hochbetagter Wursthausierer'.\cite[S.287]{pir} Daraufhin fährt er den Grenzposten an \cite[S.287]{pir} und bedroht ihn mit den Worten 'Halt die Backen, Grützkopf, sonst kannst du deine Füllung von der Straße kratzen' mit dem Tod. Im Kloster, als Mordan seine Sünden bereuen fragt er sich 'Was habe ich schon Schlimmes gesagt Pfeffer und Asche nochmal?' und würde gern die Auseinandersetzung mit dem Abt suchen, \cite[S.468]{pir} deswegen wäre '(er) gerne (...) unverzüglich aufgesprungen  (...) um diesem Gemüse zu zeigen, was eine anständige Vinaigrette ist',\cite[S.468]{pir} um ihm zu zeigen 'wo es langgeht'. Die Kleiekuchen unterhalten sich in einer 'finstere(n) Spelunke am Rand des Hafens von Murseille'\cite[S.535f]{pir} abfällig über Hörnchen indem sie ihn 'Weißbrotbengel' \cite[S.537]{pir} und 'Weißmehltölpel',\cite[S.537]{pir} nennen.


Aufwertung
Vielfach werden statt der eigentlichen Bezeichnungen für eine Person, Sache oder Angelegenheit beschönigende Ersatzausdrücke hauptsächlich zum Zweck einer Aufwertung verwendet. Dadurch kann entweder das Bezeichnete selbst mit Absicht gewürdigt oder in ein besseres Licht gestellt werden, oder es wird der damit in Zusammenhang stehenden Person oder Personengruppe Höflichkeit und Wertschätzung entgegengebracht. Solches ist etwa bei den sprachlichen Normen der Political Correctness der Fall.
Milderung und Schonung
Euphemismen können eine abschwächende Wirkung haben und das Gesagte weniger drastisch erscheinen lassen. Eine solche Verwendungsweise schont die Gefühle der angesprochenen Person oder auch des Sprechers selbst und kommt dort zum Einsatz, wo Ehrerbietung, Rücksichtnahme und Höflichkeit angebracht sind, also vorwiegend dort, wo bestimmte gesellschaftliche und kulturelle Konventionen und Normen eine solche Rücksichtnahme erforderlich machen.
Verhüllung, Tarnung und Vertuschung
Euphemismen dieser Art bezeichnen eine Sache oder einen Sachverhalt in der Weise, dass das eigentlich Gemeinte auf der Wortebene nicht – oder zumindest nicht vordergründig – in Erscheinung tritt. Dies liegt vor allem bei strengen Normen und tabuisierten Inhalten vor oder dort, wo Sachverhalte bewusst verhüllt werden sollen,[2] um beispielsweise eine öffentliche Empörung zu verhindern, so etwa im öffentlichen Sprachgebrauch totalitärer politischer Systeme, wo dieses Mittel auch ausdrücklich als sprachpolitische Maßnahme eingesetzt wird. Denn mit einer derartigen Gebrauchsweise kann letztlich eine bewusste Beeinflussung der Angesprochenen im eigenen Interesse erfolgen.					
	


Die Stilanalyse eruiert die rhetorischen bzw. stilistischen Mittel, die zum Einsatz kommen.
Stilarten: Nominalstil, Verbalstil, Adjektivstil
Verbalstil: das Gewicht liegt auf den Verben, auf dem Geschehen; er ist lebendig und natürlich.

Leitmotive Farben stimmungen symbloe personen sätze als Leitmotiv
Personifikation und Vergleiche, Ironie
Rhetorische Figuren:
lexikalische Mittel: 
Oxymoron alter Knabe, 
Alliteration, Hyperbel Übertreibung, 
Litotes Untertreibung ich ärgerte mich nicht wenig darüber, 
 Metonymie das Mittelalter glaubte verallgemeinerung nicht wörtlich übertragener sinn, 
Pleonasmus

kompositorische Mittel: 
Parenthese Einschub von Satzteil mit Gedankenstrichen!!!!!!!!!
Redeformen: wörtliche, indirekte, erlebte Rede, innerer Monolog

Stilschicht: poetisch, fachsprachlich, normalsprachlich, umgangssprachlich
Grundgestus der Sprache (auch: Stilfärbung): poetisch, gehoben, neutral, salopp, vulgär.
Letztere Differenzierung soll am Beispiel des Begriffs „sterben“ verdeutlicht werden: „ewigen Frieden gefunden“ (poetisch), „verblichen“ (gehoben), „gestorben“ (neutral), „die Mücke gemacht“ (salopp), „abgekratzt“ (vulgär).
				


\section{Einsatz im Unterricht}

\subsection{Pädagogischer Wert}
Das Kinderbuch „Die wilden Piroggenpiraten“ ist für Kinder ab 8 Jahren geeignet und bietet vielfältige Möglichkeiten im Unterricht. Durch die vielen einzelnen Kapitel fällt es, trotz der 647 Seiten, leicht, zur Entspannung nach einer anstrengenden Lernphase oder als Belohnung nach einem gelungenen Tag, ein oder mehrere Kapitel vorzulesen, ohne dabei mitten in einer Geschichte stoppen zu müssen. Auch die Abgrenzung der einzelnen Charaktere in den verschiedenen Kapiteln - ab Kapitel 4 hat immer mindestens einer der Hauptprotagonisten, ausgenommen in Kapitel 59 und 64 ein von den anderen unabhängiges Kapitel - ermöglicht nur ausgewählte Sequenzen in den Unterricht einzubauen. Die Lernbereiche im Deutschunterricht können vielfältig gefördert werden. Im Lernbereich „Sprechen“  kann aus der Lektüre vorgelesen, im Unterrichtsgespräch die Handlung zusammengefasst und in Gruppenarbeit im Text vorkommende Fachbegriffe recherchiert werden, die dann von den „Experten“ einer jeden Gruppe erklärt werden. „Sprache untersuchen“ und „Richtig schreiben“ lässt sich anhand der vielen verschiedenen Lebensmittel, die beschrieben werden. Auch unterstützt die Lektüre „Ziele für den Literaturunterricht“: Sie fördert bei richtiger Anwendung im Unterricht und im Klassenzimmer die „Lesefreude“ und beeinflusst bei gelungener Umsetzung die „Texterschließungskompetenz“. Die ungewöhnlichen Protagonisten und die ungewohnte Umgebung, sowie die detaillierte Beschreibung der Szenen und Personen, fördern „Imagination und Kreativität“. Da das Buch wegen der Wahl einer weiblichen Hauptfigur in einer eigentlich typischen Jungengeschichte kein geschlechterspezifisches Publikum anspricht und die Geschichte sich also sowohl um weibliche als auch männliche Protagonisten dreht, bietet sich für jeden Schüler die Möglichkeit sich persönlich mit einer der Figuren zu identifizieren. Auch die vielen unterschiedlichen Personen tragen zur „Identitätsfindung und (zum) Fremdverstehen“ bei und können bei der „Auseinandersetzung mit anthropologischen Grundfragen“ helfen: Was hat sich die Blutwurst gedacht  als 'sie (sich von ihren Häscherlos(riss)(...) nach ihrem Bauch griff (...) (und sich) mit einem Ruck die schwarze Pelle von einem Ende bis zum anderen auf(riss)' \cite[S.229]{pir},  warum hat sie so gehandelt?



\subsection{Beispiele von Verwendung im Unterricht gemäß des Lehrplans}

Die Lektüre kann bezogen auf den Lehrplan der Grundschule \cite{Lehrplan der Grundschule, Bayern 2000} auf vielfältige Weise in den Klassen 3 und 4 im Unterricht eingesetzt werden. 

\subsubsection{Fach Deutsch - Sprechen und Gespräche führen}

Dies sieht man zum Beispiel im Bereich Deutsch der 3. und der 4.Klasse Sprechen und Gespräche führen im Unterpunkt 3.1.1 'Einander erzählen und einander zuhören: interessant und spannend erzählen' oder 'Die Erzählperspektive wechseln' (4.Klasse) im\cite{Lehrplan der Grundschule, Bayern 2000} 
Hier können die Schüler zu Beginn einer Unterrichtseinheit, bezogen auf das Buch 'Die wilden Piroggenpiraten',\cite{pir} die Geschichte der vergangenen Stunde in eigenen Worten zusammenfassen und nachzuerzählen. Mit Techniken, die man sich in diesem Rahmen aneignet, lernt man die wichtigsten Vorgänge einer Geschichte herauszufiltern und  bereitet damit das Schreiben von Berichten in höheren Jahgangsstufen vor. 
Am Ende einer (spannenden) Sequenz bietet es sich an, die Fantasie der Schüler anzuregen und sie einen möglichen Weitergang der Geschichte erfinden zu lassen. Dabei fördert man das Leseverständnis, die Konzentration auf den Handlungsinhalt und die Rahmenhandlung. In schwächeren Klassen können dabei auch Reiz- oder Schlagworte zur Hilfe genommen werden. 
Auch können die Schüler aus der Sicht eines ausgewählten Charakters berichten oder kommentieren. Dabei wendet man die möglicherweise vorher erlernten W-Fragen an: Wie hat sich der Charakter gefühlt? An wen denkt Mohnschnecke bei der Reise auf der Speckkugel? Wie fühlt sich Eclair, als er von Mohnschnecke nach Hause geschickt wird? Wie ist es als Pirogg auf der Speckkugel zu wohnen? Warum versteckt sich Mohnschnecke nicht irgendwo, wenn sie beim Landgang die Möglichkeit dazu hat?
Wenn Schüler aus der Sicht eines ausgewählten Charakters berichten oder kommentieren, lassen sich sowohl Imagination als auch Einfühlungsvermögen und schnelle Auffassungsgabe fördern: Sie kennen die Charaktere bereits und können sich aufgrund vergangener Ereignisse erschließen wie sie wohl handeln würden. Falls sie dabei den Namen verschweigen, kann dies wiederum bei den Zuhörern das Kombinationsvermögen unterstützen.

\subsubsection{Fach Deutsch - Für sich und andere schreiben}

Bezogen auf den Lehrplan '3.2 Für sich und andere schreiben' Unterpunkt 'Zu Texten schreiben, auf Texte antworten' bietet es sich an die Grundlagen des Briefeschreiben zu erlernen und anhand der Lektüre zu üben. So können die Schüler aus der Sicht von Mohnschnecke einen Brief an ihre Eltern oder ihre beste Freundin verfassen, in dem sie schreibt wie es ihr auf der Speckkugel geht. Der Kreative Umgang mit Texten bietet dabei ein breites Spektrum an Möglichkeiten. So kann ein Tagebucheintrag verfasst und gestaltet, ein Vermisstenplakat für Mohnschnecke oder ein Kopfgeldplakat für einen der Piraten entworfen, ein Gedicht, Elfchen oder Zeitungsartikel verfasst, ein selbstgesprochenes Hörspiel entwickelt oder ein szenisches Stück geplant werden.

\subsubsection{Heimat- und Sachunterricht - Zusammenleben}

Unter 3.4.2 heißt es 'Menschen arbeiten' und darunter 'Einen Betrieb / eine Organisation in der Region erkunden - Handwerksbetrieb'. Hier bietet sich die Möglichkeit auf Grundlage des Buches Stunden zum Thema Bäckerei zu planen und diese zu besuchen. Auch eine Reederei oder das Meereskundemuseum können in diesem Zusammenhang mit eingeplant werden. So kann die Lektüre vertieft und mit einem anderen Lernbereich verknüpft werden bzw dieser darauf aufbauend verwendet werden.


http://www.lehrplanplus.bayern.de/jahrgangsstufenprofil/grundschule/3

\subsection{Unterrichtsbeispiele}



\subsubsection{Unterrichtsentwurf: Wir backen Mohnschnecken}

Die Grundvorraussetzung für diese Unterrichtseinheit ist, dass die Schüler die Hauptfigur des Romans 'Die wilden Piroggenpiraten' bereits kennengelernt und sich mit ihr befasst haben. Zudem sollte das Analysieren sachlicher Texte auf die Grundinformationen, das Befolgen einfacher Regeln und das angemessene Verhalten in der Schulküche bekannt sein. Sind letztere nicht bekannt, so kann die Stunde auch darauf ausgelegt werden, eine dieser Kompetenzen zu erlernen, sodass die Erlernung einer einzelnen Kompetenz dem jeweiligen Arbeitsschritt vorrausgeht.

Die Zutaten könnten von der Lehrkraft zur Verfügung gestellt werden, man kann das Konzept aber auch in zwei Unterrichtsteile zerlegen, wobei man zunächst Teil eins der folgenden Modellstunde, mit der Analyse des Backrezeptes behandelt und die Schüler in einer Folgestunde, die erarbeiteten und in den einzelnen Gruppen zugewiesenen Zutaten mitbringen lässt.

Es empfiehlt sich in den ersten zwei Schulstunden oder in einer der beiden Stunden nach der Pause zu beginnen, so bleibt Spielraum zur Pause oder zum Unterrichtsende, falls ein Teilbereich länger dauern sollte als geplant. Zudem können die Schüler so ihre produzierten Werke gleich danach verzehren oder sind nach der Pause nicht mehr allzu naschfreudig und können die Mohnschnecken mit nach Hause nehmen.

Ziel ist die Vertiefung der Lektüre, die Aufarbeitung der Geschichte in einem anderen Rahmen, die Förderung des gemeinsamen Arbeitens und des Befolgens von Regeln in einem bestimmten Umfeld, sowie die kreative Verknüpfung zweier unabhängiger Bereiche.

Der erste Teil der Einheit konzentriert sich auf die Analyse des Backrezeptes. Je nachdem wieviele Arbeitsplätze in der Schulküche zur Verfügung stehen, geschieht dies in Schülergruppen von drei bis fünf Personen. Eine kleinere Gruppe ist im Notfall möglich, kleinere Gruppen im Bezug auf die ganze Klasse setzen jedoch eine größere Materialmenge vorraus. Zunächst notieren sich die Schüler einen Überblick über die Zutatenliste und die einzelnen Arbeitsschritte. Gemeinsam werden Fachbegriffe geklärt, die Ergebnisse besprochen und gesichert und Unklarheiten bereinigt. Danach wird die weitere Vorgehensweise geklärt. Sollte es keine Vorverteilung der Aufgaben bei Gruppenarbeiten geben, wird diese in Hinblick auf das Arbeiten in der Schulküche geklärt.

Im zweiten Teil werden in der Schulküche die Gruppen auf ihre Arbeitsplätze verteilt und im Frontalunterricht zunächst die benötigten Geräte und Zutaten besprochen und bereitgestellt. Schritt für Schritt befolgen die Schüler in eigenständiger Arbeit nun das Rezept. Je nach Jahrgangsstufe und Lernfortschritt wird etwa in einer 3.Klasse  jeder Schritt erst besprochen, falls es Fragen oder mögliche Schwierigkeiten gibt, werden diese vorher erläutert und danach in eigenständiger Arbeit in den Gruppen ausgeführt. Eine 4.Klasse kann je nach Kompetenz bis zu bestimmten Zwischenschritten, eigenständig in der Gruppe arbeiten und ihre Vorgehensweise bei diesen erklären und begründen.

Zuletzt werden die einzelnen Schirtte rückblickend wiederholt und die Ergebnisse, die fertigen Mohnschnecken gesichtet. Dabei können Fotos, die die Lehrkraft gemacht hat, aber möglicherweise auch ein Schüler in jeder Gruppe, der die wichtigsten Schritte fotografieren sollte zur Ergebnissicherung dienen.

\subsubsection{Unterrichtsentwurf: Ich gestalte einen Fotoroman}

Das Unterrichtsthema 'Wir gestalten einen Fotoroman zum Buch 'Die wilden Piroggenpiraten' lässt sich im Lehrplan der Jahrgangsstufe 4 unter 'Bilderwelt der Medien' in Anlehnung an „4.4 Bewegte Bilder und ihre Helden' unter dem besonderen Gesichtspunkt von ''Eigene 'Helden' mit besonderen Attributen unter bewusstem Einsatz von gestalterischen Mitteln schaffen' einordnen. Den Schülergruppen wird eine beschränkte Auswahl an Kapiteln oder Seiten aus dem Buch 'Die wilden Piroggenpiraten' zur Verfügung gestellt. Daraus analysieren sie ausschlaggebende Merkmale ihrer Hauptfigur, sowie ihre Interaktion mit anderen Figuren und überlegen wie sie diese filmisch/fotografisch in Szene setzen können. 

Vorraussetzungen für diese Unterrichtseinheit sind, dass die Schüler die Hauptfiguren des Romans schon kennengelernt und sich eine gewisse Zeit im Unterricht mit der Lektüre befasst haben, damit ihnen die Charaktere, ihre Besonderheiten und zumindest ein paar ausgewählte Schauplätze bekannt sind. Zudem sollten sie  Grundinformationen aus einem Text erschließen und die Sachlage im Bezug auf Ort, Zeit und am Geschehen teilhabende Personen erarbeiten können. Von Vorteil ist, wenn die Schüler im szenischen Gestalten möglicherweise schon Mimik und Gestik von Personen in einer bestimmten Gefühlslage kennengelernt haben. Der Umgang mit einem Fotoapparat, seine Funktionsweise und Handhabung ist die wichtigste Grundvorraussetzung für die Durchführung. Sollte ein vorrausgesetzter Teilbereich im Unterricht noch nicht abgedeckt worden sein, sollte dieser vorher erlernt oder kann bei entsprechender Veränderung mit Hilfe des folgenden Modells erlernt werden.

Die Fotoapparate können von zu Hause mitgebracht werden, wobei man die Schüler darauf hinweist die Eltern um eines der älteren Modelle, wenn möglich analog, zu bitten, um sowohl größere Sachschäden an der neuesten Digitalkamera vorzubeugen, als auch die Möglichkeit zu haben die Filme tatsächlich entwickeln zu lassen. Einwegexemplare können auch von vorneherein oder im Notfall von der Lehrkraft zur Verfügung gestellt werden. 
Plakate, Papier und Kleber sind normalerweise in der Klasse vorhanden oder werden von der Lehrkraft besorgt.

Die Umsetzung sollte, zumindest beim Fotografieteil, nach der Schulpause angesetzt werden. So bleibt genug Spielraum bis zum Unterrichtsende und die Schüler haben sich in der Pause ausgetobt. arbeiten somit konzentrierter mit dem ansonsten nicht im Unterricht verwendeten Medium.

Ziele sind neben der Vertiefung der Lektüre, die Aufarbeitung der Geschichte in einem anderen Rahmen, die Förderung des Arbeitens in der Gruppe, der verantwortungsbewusste Umgang mit technischen Geräten, interaktives Lernen, der Einsatz von Medien und das Erlernen eines Bewusstseins der Medienwirkung auf sich selbst und Andere.

Im ersten Teil wird in Gruppen ein ausgewählter Abschnitt - zum Beispiel der Angriff der Speckpiroggen auf die Stadt Madeira \cite[180ff]{pir} - analysiert. Die Schüler erkennen bekannte Charaktere, beschreiben die Umgebung und erzählen kurz nach, was in diesem Teil geschieht und welche Ereignisse entscheident sind. Gemeinsam werden die Ergebnisse zusammengetragen und gesichert.  Nun wird erklärt, dass jede Gruppen einen Fotoroman zu einer Textsequenz des Kapitels erstellen darf. Dabei werden anhand eines Beispiels die wichtigsten Merkmale dieser Textgestaltung abgesprochen und der Rahmen für das später eigenständig durchzuführende Projekt gesteckt. So ist in jeder Gruppe, in Klassen mit unentschlossenen Schülern durch losen, festzustellen, wer welchen Charakter darstellt, wieviele Fotos maximal pro Szene gemacht werden dürfen und wie frei sich die Gruppen im Schulgebäude bewegen dürfen. Nach einem abgesteckten Zeitraum, der je nach Klasse 20 bis 30 Minuten dauern kann, kommen die Schüler in die Klasse zurück. Jede Gruppe erzählt kurz wie leicht oder schwer ihnen die Aufgabenumsetzung gefallen ist und erklärt exemplarisch eine Schwierige und eine leicht umzusetzende Szene. Auch Probleme mit der Handhabung der Kamera oder beim Nachahmen einer bestimmten Gefühlslage können angesprochen und in der Klasse besprochen und geklärt werden. Falls der Zeitrahmen zur Fertigstellung des ersten Teils noch zu kurz war, notieren sich die Schüler welche Szene sie zuletzt umgesetzt haben und stellen sie in einer weiteren Stunde, an einem anderen Tag fertig. Ansonsten werden die belichteten Filme von den Schülern selbst aufgewickelt bzw. der Fotoapparat zurückgespult und mit Gruppennamen versehen eingesammelt, bzw. die Einwegkameras mit Namen beschriftet und zur Lehrkraft gegeben. Diese entwickelt sie um sie dann im zweiten Teil der Stunde bearbeiten zu können.

Nach der Entwicklung der Fotos werden zunächst die Eigenschaften eines Fotoromans noch einmal ins Gedächtnis gerufen, dann erhält jede Gruppe ihre selbst angefertigten Fotos. Es kann abgesprochen werden, in der Gruppe das gelungenste, lustigste oder in irgendeiner Art auffälligste Foto herauszusuchen und dieses den anderen Klassenkameraden vorzustellen. So wird das genauere Betrachten der Bilder gefördert und allzustarke Unruhe in übermütigen Klassen vermieden. Nun kann der Fotoroman gefertigt werden. In einzelnen Schritten wird zusammen in den Gruppen gearbeitet. Erst werden die Fotos in eine logische Reihenfolge gebracht, dann die entsprechenden Erzähl- und Sprechblasen gefertigt. Schließlich wird das erarbeitete Konzept auf ein Plakat geklebt und gemeinsam den anderen Gruppen präsentiert, angefangen bei der Gruppe, die den Anfang des Kapitels bearbeitet hat usw. Erneut kann zur Vorbereitung zukünftiger Projekte besprochen werden, welche Schritte besondere Schwierigkeiten bereitet haben und was den Schülern besonders leicht gefallen ist. So werden lernen sie auch in der Gruppe Schäche zeigen zu können, dass auch andere dasselbe Problem haben und, dass man gemeinsam zu einer Problemlösung kommen kann. Zur Gestaltung des Schulhauses können die Plakate dann auf dem Flur aufgehängt werden.


\chapter{Erläuterung zum Gebrauch im Unterricht}